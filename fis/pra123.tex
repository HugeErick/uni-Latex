\documentclass{article}
%\usepackage[spanish,activeacute]{babel}
%\usepackage[english,activeacute]{babel}
%\usepackage[latin1]{inputenc}
\usepackage[utf8]{inputenc}
\usepackage[english]{babel}

\usepackage{amsmath,amsfonts,amssymb,amstext,amsthm,amscd}
\usepackage{hyperref}
\usepackage{latexsym}
\usepackage{graphicx}
%\usepackage{subfigure}
\usepackage{subfig}
%\linespread{1.6}
\usepackage{float}
\usepackage{dcolumn}% Align table columns on decimal point(esto lo saque del ejemplo de revtex4)
\usepackage{bm}% bold math(esto lo saque del ejemplo de revtex4)
\newcounter{itemR}
\usepackage{here} %recordar usar el comando[H] para las gráficas que es el comando here en lugar de [h!]
\usepackage{fancyhdr}
%\usepackage{sidecap}
%\usepackage[spanish,activeacute]{babel}
\usepackage{multirow}
\usepackage{multicol}
\usepackage{array}
\usepackage{enumitem}
%\usepackage{booktabs}% para hacer tablas profesionales con \toprule

% ------------------------------------------------------------------------------------------------------------------------------------------------------

\usepackage{fancyhdr}
\setlength{\headheight}{15.2pt}
\usepackage[paperwidth=8.5in, paperheight=11.0in, top=1.0in, bottom=1.0in, left=1.0in, right=1.0in]{geometry}

\pagestyle{fancyplain}
\fancyhead[LE,RO]{Práctica $\#$1 \& 2, Faraday, fuerza electrica}
\fancyhead[CE,CO]{}
\fancyhead[RE,LO]{P23-FIS1012-12}
\fancyfoot[LE,RO]{\thepage}
\fancyfoot[CE,CO]{Laboratorio de Física, UDLAP}
\fancyfoot[RE,LO]{}

% ------------------------------------------------------------------------------------------------------------------------------------------------------
% ------------------------------------------------------------------------------------------------------------------------------------------------------
% ------------------------------------------------------------------------------------------------------------------------------------------------------

\begin{document}

\fancypagestyle{plain}{
   	\renewcommand{\headrulewidth}{1pt}
   	\renewcommand{\footrulewidth}{1pt}
}

\renewcommand{\footrulewidth}{1pt}
\renewcommand{\tablename}{Tabla}
\renewcommand{\figurename}{Figura}

% ------------------------------------------------------------------------------------------------------------------------------------------------------
% ------------------------------------------------------------------------------------------------------------------------------------------------------
% ------------------------------------------------------------------------------------------------------------------------------------------------------

\title{Carga y fuerza eléctrica}
\author{\small{Luis Alberto Gil Bocanegra ID: 177410, Erick Gonzalez Parada ID: 178145}\\
 \small{Gartzen Aldecoa Barroso ID: 178034 .}\\		% ----- Varios autores separarlos por comas:  \small{Nombre(s) de (los) autor(es)\footnote{ID; correo@udlap.mx}, Nombre(s) de (los) autor(es)\footnote{ID; correo@udlap.mx}
	   \small{Depto. de Actuaría, Física y Matemáticas, Universidad de las Américas Puebla, Puebla, M\'exico 72810}}
\date{\small{\today}}

\maketitle

% ------------------------------------------------------------------------------------------------------------------------------------------------------
% ------------------------------------------------------------------------------------------------------------------------------------------------------
% ------------------------------------------------------------------------------------------------------------------------------------------------------

\begin{abstract}
	Durante las 3 prácticas se observa...	
\\
\\
{\it Keywords:}  campo, electricidad  
\\
\\
\end{abstract}

% ------------------------------------------------------------------------------------------------------------------------------------------------------

\begin{multicols}{2}

\section{Desarrollo teórico}\label{Desarrollo Teorico}                              	% -------------------- Introducción
\subsection{Objetivo primera práctica}\label{Objetivo primera práctica}
	Observar la existencia y tipos de carga
	\subsection{marco teórico \ref{Objetivo primera práctica}}
	\paragraph*{Carga eléctrica}
	La carga eléctrica es una propiedad fundamental de la materia que se manifiesta debido 
	a la interacción de partículas subatómicas, como electrones (\ensuremath{-e^{-}}) y protones 
	(\ensuremath{+p^{+}}). Los electrones tienen una carga eléctrica negativa
	(\ensuremath{q_{e}=-1.602\times 10^{-19}\mathrm{C}}), mientras que los protones
	tienen una carga eléctrica positiva (\ensuremath{q_{p}=1.602\times 10^{-19}\mathrm{C}}).
	La carga eléctrica es una propiedad cuantizada, lo que significa que solo puede existir 
	en múltiplos enteros de la carga elemental.

	\paragraph*{Tipos de carga eléctrica}
\begin{enumerate}
\item Carga Positiva: Producida por protones (\ensuremath{+p^{+}}), tiene una polaridad positiva (\ensuremath{q>0}).
\item Carga Negativa: Producida por electrones (\ensuremath{-e^{-}}), tiene una polaridad negativa (\ensuremath{q<0}).
\item Carga Neutra: Un objeto es eléctricamente neutro cuando tiene un número igual de protones y electrones, lo que resulta en una carga neta de cero (\ensuremath{q=0}).
\item Carga Ionizada: Un átomo o molécula puede ganar (\ensuremath{q>0}) o perder (\ensuremath{q<0}) electrones, formando iones con carga eléctrica.
\item Carga Cuantizada: La carga eléctrica viene en múltiplos de la carga elemental (\ensuremath{q_{e}=1.602\times 10^{-19}~\mathrm{C}}).
\end{enumerate}
\subsection{Objetivo segunda práctica}\label{Objetivo segunda práctica}	
	Determinar la carga, observar los dos tipos de carga, inducción y depósito de carga.
	\subsection{marco teórico \ref{Objetivo segunda práctica}}
	\paragraph*{inducción y depósito de carga}
	La inducción eléctrica se refiere al proceso mediante el cual se redistribuyen
	las cargas eléctricas en un objeto debido a la influencia de un campo eléctrico externo,
	sin que haya contacto físico directo con otro objeto cargado. A continuación, se describen
	los conceptos clave:

	\begin{enumerate}
		\item Carga Inducida: Cuando un objeto cargado se acerca a un objeto neutro, el campo eléctrico del primero ejerce una fuerza sobre las cargas en el segundo objeto, causando una redistribución de las cargas en la superficie del objeto neutro.

\item Inducción Electrostática: La inducción electrostática es un fenómeno en el que un objeto cargado, conocido como inductor, causa la redistribución de cargas en otro objeto cercano sin contacto directo.

\end{enumerate}

\section{Depósito de Carga Eléctrica}

\begin{enumerate}

\item Conducción: En la conducción, la carga eléctrica se transfiere de un objeto cargado a otro mediante contacto directo.

\item Fricción: En algunos casos, la carga eléctrica puede transferirse de un objeto a otro debido a la fricción entre los materiales.

\item Inducción por Contacto: En esta técnica, un objeto cargado se acerca a un objeto neutro y se establece un contacto momentáneo.

\item Inducción por Frotamiento: Al frotar dos objetos aislantes entre sí, como un globo de goma y una lana, se pueden transferir cargas eléctricas de un objeto al otro debido a la fricción.

\end{enumerate}

El entendimiento de la inducción y el depósito de carga eléctrica es esencial
en la electrostática y es fundamental en la comprensión de la forma en que los
objetos pueden adquirir cargas eléctricas y cómo interactúan en presencia de campos eléctricos.

\section{Desarrollo Experimental}\label{Desarrollo experimental}				% -------------------- Metodología 
\subsection{práctica 1}\label{p1}

	Vidrio y seda atrae al acrílico\\ 
	Vidrio y piel de conejo atrae al acrílico\\ 
	Vidrio y fieltro atrae muy poco al acrílico\\ 

	Seda y vidrio atrae al pbc\\
	piel de conejo y acrílico atrae muy poco al pbc\\
	acrílico y seda atrae muy poco al pbc\\
	vidrio y fieltro atrae al pbc\\
	fieltro y acrílico no atrae al pbc\\
	piel de conejo y vidrio atrae al pbc\\ \\
\subsection{práctica 2}\label{p2}

	sin tocar la jaula \\
	vidrio con piel de conejo es positivo \\
	vidrio con seda es positivo \\
	vidrio con fieltro es negativo \\
	pbc con fieltro es negativo \\
	pbc con piel de conejo negativo \\
	pbc con seda es positivo \\
	acrílico con fieltro es positivo \\
	acrílico con piel es negativo \\
	acrílico con seda es positivo \\
	\\
	tocando la jaula interna \\
	vidrio con piel de conejo es positivo \\
	vidrio con seda es positivo \\
	vidrio con fieltro es positivo \\
	pbc con fieltro es negativo \\
	pbc con piel de conejo negativo \\
	pbc con seda es positivo \\
	acrílico con fieltro es positivo \\
	acrílico con seda es positivo \\
	acrílico con piel es negativo \\
	\\
	cuando lo conectas con tierra le quitas lo que tenia forzando un neutro de tal manera que ahora le falta
	vidrio con piel de conejo de negativo se anula y se va a positivo\\
	acrílico con piel de conejo va de positivo a negativo \\
	pbs con piel de conejo va de negativo a positivo \\

\end{multicols}
\section{Resultados y análisis}\label{Resultados}			% -------------------- Resultados

\section{Conclusiones}\label{Conclusiones}				% -------------------- Conclusiones

\begin{thebibliography}{9}						% -------------------- Bibliografía
	\bibitem{Martín}
		Martín, I. (2004). Física General
		\bibitem{Serway}
		Serway, R. A., $\&$ Jewett, J. W. (2008). Física para ciencias e ingeniería. (7.a
ed., Vol. 1). CENGAGE Learning.

\bibitem{Pérez}
	Newton, I. (1687). Philosophiæ Naturalis Principia Mathematica [Mathematical Principles of Natural Philosophy]. Londini: Jussu Societatis Regiæ ac Typis Josephi Streater.

\bibitem{Khan Academy}
	Anonymous. (2017). ¿Qué es la fricción? based on \cite{Openstax College Physics}. "from https://es.khanacademy.org/science/physics/forces-newtons-laws/inclined-planes-friction/a/what-is-friction"
	\bibitem{Openstax College Physics}
		Openstax College Physics. (n.d). Friction. from http://cnx.org/contents/031da8d3-b525-429c-80cf-6c8ed997733a@9,4:32/Friction"

\end{thebibliography}
\end{document}	