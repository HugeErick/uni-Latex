\documentclass{article}
%\usepackage[spanish,activeacute]{babel}
%\usepackage[english,activeacute]{babel}
%\usepackage[latin1]{inputenc}
\usepackage[utf8]{inputenc}
\usepackage[english]{babel}

\usepackage{amsmath,amsfonts,amssymb,amstext,amsthm,amscd}
\usepackage{hyperref}
\usepackage{latexsym}
\usepackage{graphicx}
%\usepackage{subfigure}
\usepackage{subfig}
%\linespread{1.6}
\usepackage{float}
\usepackage{dcolumn}% Align table columns on decimal point(esto lo saque del ejemplo de revtex4)
\usepackage{bm}% bold math(esto lo saque del ejemplo de revtex4)
\newcounter{itemR}
\usepackage{here} %recordar usar el comando[H] para las gráficas que es el comando here en lugar de [h!]
\usepackage{fancyhdr}
%\usepackage{sidecap}
%\usepackage[spanish,activeacute]{babel}
\usepackage{multirow}
\usepackage{multicol}
\usepackage{array}
\usepackage{enumitem}
%\usepackage{booktabs}% para hacer tablas profesionales con \toprule

% ------------------------------------------------------------------------------------------------------------------------------------------------------

\usepackage{fancyhdr}
\setlength{\headheight}{15.2pt}
\usepackage[paperwidth=8.5in, paperheight=11.0in, top=1.0in, bottom=1.0in, left=1.0in, right=1.0in]{geometry}

\pagestyle{fancyplain}
\fancyhead[LE,RO]{Práctica $\#$9, colisiones}
\fancyhead[CE,CO]{}
\fancyhead[RE,LO]{P23-FIS1012-12}
\fancyfoot[LE,RO]{\thepage}
\fancyfoot[CE,CO]{Laboratorio de Física, UDLAP}
\fancyfoot[RE,LO]{}

% ------------------------------------------------------------------------------------------------------------------------------------------------------
% ------------------------------------------------------------------------------------------------------------------------------------------------------
% ------------------------------------------------------------------------------------------------------------------------------------------------------

\begin{document}

\fancypagestyle{plain}{
   	\renewcommand{\headrulewidth}{1pt}
   	\renewcommand{\footrulewidth}{1pt}
}

\renewcommand{\footrulewidth}{1pt}
\renewcommand{\tablename}{Tabla}
\renewcommand{\figurename}{Figura}

% ------------------------------------------------------------------------------------------------------------------------------------------------------
% ------------------------------------------------------------------------------------------------------------------------------------------------------
% ------------------------------------------------------------------------------------------------------------------------------------------------------

\title{carga y fuerza eléctrica}
\author{\small{Luis Alberto Gil Bocanegra ID: 177410, Erick Gonzalez Parada ID: 178145}\\
 \small{Gartzen Aldecoa Barroso ID: 178034 .}\\		% ----- Varios autores separarlos por comas:  \small{Nombre(s) de (los) autor(es)\footnote{ID; correo@udlap.mx}, Nombre(s) de (los) autor(es)\footnote{ID; correo@udlap.mx}
	   \small{Depto. de Actuaría, Física y Matemáticas, Universidad de las Américas Puebla, Puebla, M\'exico 72810}}
\date{\small{\today}}

\maketitle

% ------------------------------------------------------------------------------------------------------------------------------------------------------
% ------------------------------------------------------------------------------------------------------------------------------------------------------
% ------------------------------------------------------------------------------------------------------------------------------------------------------

\begin{abstract}
	Durante las 3 prácticas se observa...	
\\
\\
{\it Keywords:}  campo, electricidad  
\\
\\
\end{abstract}

% ------------------------------------------------------------------------------------------------------------------------------------------------------

\begin{multicols}{2}

\section{Desarrollo teórico}\label{Desarrollo Teorico}                              	% -------------------- Introducción
\subsection*{Objetivo primera práctica}\label{Objetivo primera práctica}
	Observar la existencia y tipos de carga
\subsection*{Objetivo segunda práctica}\label{Objetivo segunda práctica}	
	Determinar la carga, observar los dos tipos de carga, inducción y depósito de carga.
\subsection*{Objetivo tercera práctica}\label{Objetivo tercera práctica}	
	Determinar la carga depositada en una esfera conductora, en un campo eléctrico constante.


\section{Desarrollo Experimental}\label{Desarrollo experimental}				% -------------------- Metodología 

\section{temp}\label{temp}
	práctica 1 \\
	Vidrio y seda atrae al acrílico\\ 
	Vidrio y piel de conejo atrae al acrílico\\ 
	Vidrio y fieltro atrae muy poco al acrílico\\ 

	Seda y vidrio atrae al pbc\\
	piel de conejo y acrílico atrae muy poco al pbc\\
	acrílico y seda atrae muy poco al pbc\\
	vidrio y fieltro atrae al pbc\\
	fieltro y acrílico no atrae al pbc\\
	piel de conejo y vidrio atrae al pbc\\ \\
	práctica 2 \\

	sin tocar la jaula \\
	vidrio con piel de conejo es positivo \\
	vidrio con seda es positivo \\
	vidrio con fieltro es negativo \\
	pbc con fieltro es negativo \\
	pbc con piel de conejo negativo \\
	pbc con seda es positivo \\
	acrílico con fieltro es positivo \\
	acrílico con piel es negativo \\
	acrílico con seda es positivo \\
	\\
	tocando la jaula interna \\
	vidrio con piel de conejo es positivo \\
	vidrio con seda es positivo \\
	vidrio con fieltro es positivo \\
	pbc con fieltro es negativo \\
	pbc con piel de conejo negativo \\
	pbc con seda es positivo \\
	acrílico con fieltro es positivo \\
	acrílico con seda es positivo \\
	acrílico con piel es negativo \\
	\\
	cuando lo conectas con tierra le quitas lo que tenia forzando un neutro de tal manera que ahora le falta
	vidrio con piel de conejo de negativo se anula y se va a positivo\\
	acrílico con piel de conejo va de positivo a negativo \\
	pbs con piel de conejo va de negativo a positivo \\

\section{Practica fuerza electricidad}\label{Practica fuerza electricidad}


\end{multicols}
\section{Resultados y análisis}\label{Resultados}			% -------------------- Resultados

\section{Conclusiones}\label{Conclusiones}				% -------------------- Conclusiones

\begin{thebibliography}{9}						% -------------------- Bibliografía
	\bibitem{Martín}
		Martín, I. (2004). Física General
		\bibitem{Serway}
		Serway, R. A., $\&$ Jewett, J. W. (2008). Física para ciencias e ingeniería. (7.a
ed., Vol. 1). CENGAGE Learning.

\bibitem{Pérez}
	Newton, I. (1687). Philosophiæ Naturalis Principia Mathematica [Mathematical Principles of Natural Philosophy]. Londini: Jussu Societatis Regiæ ac Typis Josephi Streater.
\end{thebibliography}

\end{document}	