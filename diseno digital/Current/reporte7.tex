\documentclass{article}
%\usepackage[spanish,activeacute]{babel}
%\usepackage[english,activeacute]{babel}
%\usepackage[latin1]{inputenc}
\usepackage[utf8]{inputenc}
\usepackage[english]{babel}
\usepackage{amsmath,amsfonts,amssymb,amstext,amsthm,amscd}
\usepackage{hyperref}
\usepackage{latexsym}
\usepackage{graphicx}
%\usepackage{subfigure}
\usepackage{subfig}
%\linespread{1.6}
\usepackage{float}
\usepackage{dcolumn}% Align table columns on decimal point(esto lo saque del ejemplo de revtex4)
\usepackage{bm}% bold math(esto lo saque del ejemplo de revtex4)
\newcounter{itemR}
\usepackage{here} \usepackage{fancyhdr}
%\usepackage{sidecap}
%\usepackage[spanish,activeacute]{babel}
\usepackage{multirow}
\usepackage{multicol}
\usepackage{array}
\usepackage{ragged2e}
%\usepackage{booktabs}% para hacer tablas profesionales con \toprule

% ------------------------------------------------------------------------------------------------------------------------------------------------------

\usepackage{fancyhdr}
\setlength{\headheight}{15.2pt}
\usepackage[paperwidth=8.5in, paperheight=11.0in, top=1.0in, bottom=1.0in, left=1.0in, right=1.0in]{geometry}

\pagestyle{fancyplain}
\fancyhead[LE,RO]{Reporte $\#$7}
\fancyhead[CE,CO]{}
\fancyhead[RE,LO]{P23-LRT2022-3}
\fancyfoot[LE,RO]{\thepage}
\fancyfoot[CE,CO]{Diseño digital, UDLAP}
\fancyfoot[RE,LO]{}

% ------------------------------------------------------------------------------------------------------------------------------------------------------
% ------------------------------------------------------------------------------------------------------------------------------------------------------
% ------------------------------------------------------------------------------------------------------------------------------------------------------

\begin{document}

\fancypagestyle{plain}{
   	\renewcommand{\headrulewidth}{1pt}
   	\renewcommand{\footrulewidth}{1pt}
}

\renewcommand{\footrulewidth}{1pt}
\renewcommand{\tablename}{Tabla}
\renewcommand{\figurename}{Figura}

% ------------------------------------------------------------------------------------------------------------------------------------------------------
% ------------------------------------------------------------------------------------------------------------------------------------------------------
% ------------------------------------------------------------------------------------------------------------------------------------------------------

\title{Circuitos digitales básicos}
\author{\small{Erick Gonzalez Parada ID: 178145 $\&$ Omar Martínez López ID: 177465}\\ 
\small{ Depto. Computación Electrónica y Mecatrónica.} \\
\small {Docente: Dr. Juan Carlos Moreno Rodríguez}}

\date{\small{\today}}

\maketitle

% ------------------------------------------------------------------------------------------------------------------------------------------------------

\begin{abstract}
	\begin{justify}
		Se diseño la aplicación de 3 entradas (w,x,y) y dos salidas: F(x) = verdadero cuando es menor que 5 e impar $\&$ G(x) = verdadero cuando es par y mayor que 5.	
		\end{justify}
{\it Keywords:}   diseño de hardware, álgebra booleana 
\end{abstract}
\begin{multicols}{2}
\section{Objetivo}\label{Objetivo}
El estudiante pone en práctica el procedimiento de diseño de circuitos combinacionales para generar un circuito a partir de descripciones verbales.
\section{Introducción}\label{sec:intro}
En la actualidad muchos de los aparatos electrónicos con los convivimos día con día funcionan gracias a que en su interior se encuentran las instrucciones de los procesos que estos deben realizar dependiendo de cada dispositivo por su hardware y propósito, y es así que son programados para que cumplan sus funciones, la siguiente practicas es un reforzamiento e las practicas anteriores, solamente que haciéndolo un poco más complejo para cumplir una función muy específica. Lo que llevaremos en práctica es la implementación de dos funciones que generan dos salidas distintas, cada una cumpliendo con características específicas para el encendido de un led, pero compartiendo las variables de entrada. 
\section{Análisis Teórico}\label{sec:analiTeorico}
La álgebra booleana es importante en el diseño de hardware porque permite a los diseñadores de circuitos manipular las señales binarias de entrada para obtener las señales binarias de salida deseadas. La teoría de la álgebra booleana proporciona una serie de reglas y operaciones que se pueden utilizar para simplificar y reducir las expresiones booleanas, lo que facilita el diseño y la implementación de circuitos lógicos complejos.
\section{Resultados esperados}\label{sec:resEsperados}

\begin{table}[H]
	\centering
	\begin{tabular}{|c|c|c|c|c|c|c|}
		\hline
		z & w & x & y & F(x) & G(x) &number\\
		\hline
		0 & 0 & 0 & 0 & 0 & 0 & 0 \\
		\hline
		0 & 0 & 0 & 1 & 1 & 0 & 1 \\
		\hline
		0 & 0 & 1 & 0 & 0 & 0 & 2 \\
		\hline
		0 & 0 & 1 & 1 & 1 & 0 & 3 \\
		\hline
		0 & 1 & 0 & 0 & 0 & 0 & 4 \\ 
		\hline
		0 & 1 & 0 & 1 & 0 & 0 & 5 \\ %5
		\hline
	  0 & 1 & 1 & 0 & 0 & 1 &6 \\
		\hline
		0 & 1 & 1 & 1 & 0 & 0 & 7 \\
		\hline
		1 & 0 & 0 & 0 & 0 & 1 & 8 \\
		\hline
		1 & 0 & 0 & 1 & 1 & 0 & 9 \\
		\hline
		1 & 0 & 1 & 0 & 0 & 1 & 10 \\
		\hline
		1 & 0 & 1 & 1 & 1 & 0 & 11 \\
		\hline
		1 & 1 & 0 & 0 & 0 & 1 & 12 \\ 
		\hline
		1 & 1 & 0 & 1 & 0 & 0 & 13 \\
		\hline
	  1 & 1 & 1 & 0 & 0 & 1 & 14 \\
		\hline
		1 & 1 & 1 & 1 & 0 & 0 & 15 \\
		\hline
	\end{tabular}
	\caption{Tabla de verdad a corresponder con los resultados}
	\label{tab:1}
\end{table}
\section{Resultados obtenidos}\label{sec:resObtenidos}
La siguiente evidencia muestra que se lograron los resultados esperados:

\begin{figure}[H]
	\centering	
	\includegraphics[scale = 0.1]{../../static/df1.jpeg}
	\caption{numero 3 impar para F(x)}
	\label{fig:1}
\end{figure}

\begin{figure}[H]
	\centering	
	\includegraphics[scale = 0.1]{../../static/df2.jpeg}
	\caption{numero 5, invalida para ambas funciones (i.e. no verdadera)}
	\label{fig:2}
\end{figure}

\begin{figure}[H]
	\centering	
	\includegraphics[scale = 0.1]{../../static/df3.jpeg}
	\caption{numero 6 valida para G(x)}
	\label{fig:3}
\end{figure}

\begin{figure}[H]
	\centering	
	\includegraphics[scale = 0.1]{../../static/df4.jpeg}
	\caption{numero 7 invalida para G(x)}
	\label{fig:4}
\end{figure}



\section{Conclusiones}\label{sec:conclusion}
El resultado de esta práctica es el reflejo de los aprendizajes aprendidos anteriormente, y ahora son mostrados con unos resultados más fáciles de percibir en nuestro día a día, la combinación de dos funciones que comparten las misma estrada es muy utilizada en todo tipo de aparatos y sistemas, y lo único complicado es tener bien en cuenta cuales son las distintas variables a utilizar en cada función y que concuerden, además de hacer adecuadamente el la indicación de cual es cada switch de entrada, para que no haya complicaciones.
\section*{Referencias}\label{sec:referencias}	
No hubo usos de referencias para este trabajo
\end{multicols}
\end{document}

