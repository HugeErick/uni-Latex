\documentclass{article}
%\usepackage[spanish,activeacute]{babel}
%\usepackage[english,activeacute]{babel}
%\usepackage[latin1]{inputenc}
\usepackage[utf8]{inputenc}
\usepackage[english]{babel}

\usepackage{amsmath,amsfonts,amssymb,amstext,amsthm,amscd}
\usepackage{hyperref}
\usepackage{latexsym}
\usepackage{graphicx}
%\usepackage{subfigure}
\usepackage{subfig}
%\linespread{1.6}
\usepackage{float}
\usepackage{dcolumn}% Align table columns on decimal point(esto lo saque del ejemplo de revtex4)
\usepackage{bm}% bold math(esto lo saque del ejemplo de revtex4)
\newcounter{itemR}
\usepackage{here} %recordar usar el comandao[H] Para las  graficas que es el comando here en lugar de [h!]
\usepackage{fancyhdr}
%\usepackage{sidecap}
%\usepackage[spanish,activeacute]{babel}
\usepackage{multirow}
\usepackage{multicol}
%\usepackage{array}
%\usepackage{booktabs}% para hacer tablas profesionales con \toprule

% ------------------------------------------------------------------------------------------------------------------------------------------------------

\usepackage{fancyhdr}
\setlength{\headheight}{15.2pt}
\usepackage[paperwidth=8.5in, paperheight=11.0in, top=1.0in, bottom=1.0in, left=1.0in, right=1.0in]{geometry}

\pagestyle{fancyplain}
\fancyhead[LE,RO]{Reporte $\#$4}
\fancyhead[CE,CO]{}
\fancyhead[RE,LO]{P23-LRT2022-3}
\fancyfoot[LE,RO]{\thepage}
\fancyfoot[CE,CO]{Diseño digital, UDLAP}
\fancyfoot[RE,LO]{}

% ------------------------------------------------------------------------------------------------------------------------------------------------------
% ------------------------------------------------------------------------------------------------------------------------------------------------------
% ------------------------------------------------------------------------------------------------------------------------------------------------------

\begin{document}

\fancypagestyle{plain}{
   	\renewcommand{\headrulewidth}{1pt}
   	\renewcommand{\footrulewidth}{1pt}
}

\renewcommand{\footrulewidth}{1pt}
\renewcommand{\tablename}{Tabla}
\renewcommand{\figurename}{Figura}

% ------------------------------------------------------------------------------------------------------------------------------------------------------
% ------------------------------------------------------------------------------------------------------------------------------------------------------
% ------------------------------------------------------------------------------------------------------------------------------------------------------

\title{Circuitos digitales básicos}
\author{\small{Erick Gonzalez Parada ID: 178145 $\&$ Omar Martínez López ID: 177465} 
\\ \small{Diseño digital}
\\ \small{Depto. Computación Electrónica y Mecatrónica.}
\\ \small{Docente: Dr. Juan Carlos Moreno Rodríguez}}
     %\small{Docente: Dr. Juan Carlos Moreno Rodríguez}}\\
\date{\small{\today}}

\maketitle

% ------------------------------------------------------------------------------------------------------------------------------------------------------

\begin{abstract}
El experimento consistio de montar en el protoboard la función x + y'z simplificada para comprobar sus salidas y que sea verdadero.
\\
\\
{\it Keywords:}  simplificación, integrados  
\\
\\
\end{abstract}

\begin{multicols}{2}

\section*{Objetivo}\label{Objetivo}

Nuestro objetivo fue verificar el funcionamiento de las funciones lógicas basadas en compuertas lógicas básicas AND OR y NOT.

\section*{Introducción}\label{Introducción}

Todos los dispositivos electrónicos actuales, que cuenten con una placa para procesar datos, trabajan utilizando el sistema binario, y tienen que hacer todas sus funciones con sólo  ceros y unos, es por ello que estos dígitos son manipulados para que cada aparato cumpla su función, pero para esto es necesario manipular los ceros y unos con compuertas lógicas que permitan modificar la señal original, aunque muchas veces, al buscar cierto resultado se necesitan de muchas compuertas lógicas innecesarias, que solo empeoran un producto, haciéndolo más grande y caro de producir, y es por ello que siempre es importante buscar optimizar el uso de estas compuertas lógicas, ya optimizando esto, se obtiene un mejor producto en todos los aspectos. 
Nuestra solución fue comprobar la tabla de verdad de la función simplificada por medio del protoboard.

\section*{Análisis Teórico}\label{Análisis Teórico}
Gracias a las tablas de verdad y los métodos de los mapas de Karnaugh entre otras opciones que se tiene para simplificar una función booleana y de ahí en adelante realizamos en logisim nuestro diagrama de compuertas lógica.
\begin{figure}[H]
	\centering
	\includegraphics[scale=0.45]{../static/logisim4.png}
	\caption{Diagrama de logisim}
	\label{fig:1}
\end{figure}
\section*{Procedimiento}\label{Procedimiento}
Primero que nada se tuvo que encontrar nuestra función simplificada usando el método algebraico y comprobarla con la tabla de verdad (ver tabla 1). 
Después construimos el siguiente circuito en el protoboard ver figura \ref{fig:2}.
\\
\\
Montamos el sistema asegurándonos que los integrados estuviesen correctos, puenteamos la electricidad de tal manera que todo el protoboard tuviese corriente y por ultimo empezamos a comprobar que las salidas fuesen las correctas y lógicas a nuestro función dada.
\\
\\

\section*{Resultados esperados}\label{Resultados esperados}
Como se muestra en la tabla de verdad (figura 2) los resultados de nuestra función simplificada x + y'z arrojan un 1 cuando tenemos los valores 1, 4, 5, 6, y 7
\begin{figure}[H]
	\centering
	\includegraphics[scale=0.4]{../static/tabladeverdad1.png}
	\caption{tabla de verdad}
	\label{table:1}
\end{figure}

\section*{Resultados obtenidos}\label{Resultados obtenidos}

\begin{figure}[H]
	\centering
	\includegraphics[scale=0.18]{../static/000.jpeg}
	\caption{000}
	\label{fig:2}
\end{figure}
Cabe aclarar que nuestro led enciende cuando recibe 0 volts, es decir de nuevo estamos usando la lógica inversa para que el circuito planteado tuviera sentido y verdad.
Como se observa en la figura \ref*{fig:2}, nuestro led esta encendido por lo que el mintermino F(0) no produce una salida verdadera, es decir, nos da un 0. 

\begin{figure}[H]
	\centering
	\includegraphics[scale=0.18]{../static/001.jpeg}
	\caption{001}
	\label{fig:3}
\end{figure}
	
Ahora se observa en la figura \ref*{fig:3} como esta apagado el led cuando tenemos el mintermino 1 el cual si produce una salida de 1.

\begin{figure}[H]
	\centering
	\includegraphics[scale=0.18]{../static/011.jpeg}
	\caption{011}
	\label{fig:4}
\end{figure}

\begin{figure}[H]
	\centering
	\includegraphics[scale=0.18]{../static/100.jpeg}
	\caption{100}
	\label{fig:5}
\end{figure}

\begin{figure}[H]
	\centering
	\includegraphics[scale=0.18]{../static/101.jpeg}
	\caption{101}
	\label{fig:6}
\end{figure}

\begin{figure}[H]
	\centering
	\includegraphics[scale=0.18]{../static/110.jpeg}
	\caption{110}
	\label{fig:7}
\end{figure}

Como se observa en todas las figuras los resultados obtenidos son los mismos que los resultados esperados debido a que construimos el circuito correctamente.

\section*{Conclusiones}\label{Conclusiones}
Como vemos, la simplificación de las funciones ayuda a que los circuitos sean mejores en todos los aspectos, ya que los simplifican y lo cual lo hace más sencillo de entender, y en general para las empresas y productos es mejor, ya que sus costes de producción se minimizan y los productos pueden ser más pequeños, y todo esto sin modificar el resultado obtenido en la función, sin duda el ver la función estándar y después su versión simplificada vemos que la sencillez hace que todo sea mejor y más óptimo.
\section*{Referencias}\label{Referencias}	
Brunete A. (s.f), Mapa de Karnaugh.
\href{https://bookdown.org/alberto_brunete/intro_automatica/mapa-de-karnaugh.html}{link}

\end{multicols}



\end{document}

