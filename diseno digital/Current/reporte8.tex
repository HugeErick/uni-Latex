\documentclass{article}
%\usepackage[spanish,activeacute]{babel}
%\usepackage[english,activeacute]{babel}
%\usepackage[latin1]{inputenc}
\usepackage[utf8]{inputenc}
\usepackage[english]{babel}
\usepackage{amsmath,amsfonts,amssymb,amstext,amsthm,amscd}
\usepackage{hyperref}
\usepackage{latexsym}
\usepackage{graphicx}
%\usepackage{subfigure}
\usepackage{subfig}
%\linespread{1.6}
\usepackage{float}
\usepackage{dcolumn}% Align table columns on decimal point(esto lo saque del ejemplo de revtex4)
\usepackage{bm}% bold math(esto lo saque del ejemplo de revtex4)
\newcounter{itemR}
\usepackage{here} \usepackage{fancyhdr}
%\usepackage{sidecap}
%\usepackage[spanish,activeacute]{babel}
\usepackage{multirow}
\usepackage{multicol}
\usepackage{array}
\usepackage{ragged2e}
%\usepackage{booktabs}% para hacer tablas profesionales con \toprule

% ------------------------------------------------------------------------------------------------------------------------------------------------------

\usepackage{fancyhdr}
\setlength{\headheight}{15.2pt}
\usepackage[paperwidth=8.5in, paperheight=11.0in, top=1.0in, bottom=1.0in, left=1.0in, right=1.0in]{geometry}

\pagestyle{fancyplain}
\fancyhead[LE,RO]{Reporte $\#$8}
\fancyhead[CE,CO]{}
\fancyhead[RE,LO]{P23-LRT2022-3}
\fancyfoot[LE,RO]{\thepage}
\fancyfoot[CE,CO]{Diseño digital, UDLAP}
\fancyfoot[RE,LO]{}

% ------------------------------------------------------------------------------------------------------------------------------------------------------
% ------------------------------------------------------------------------------------------------------------------------------------------------------
% ------------------------------------------------------------------------------------------------------------------------------------------------------

\begin{document}

\fancypagestyle{plain}{
   	\renewcommand{\headrulewidth}{1pt}
   	\renewcommand{\footrulewidth}{1pt}
}

\renewcommand{\footrulewidth}{1pt}
\renewcommand{\tablename}{Tabla}
\renewcommand{\figurename}{Figura}

% ------------------------------------------------------------------------------------------------------------------------------------------------------
% ------------------------------------------------------------------------------------------------------------------------------------------------------
% ------------------------------------------------------------------------------------------------------------------------------------------------------

\title{Circuitos digitales básicos}
\author{\small{Erick Gonzalez Parada ID: 178145 $\&$ Omar Martínez López ID: 177465}\\ 
\small{ Depto. Computación Electrónica y Mecatrónica.} \\
\small {Docente: Dr. Juan Carlos Moreno Rodríguez}}

\date{\small{\today}}

\maketitle

% ------------------------------------------------------------------------------------------------------------------------------------------------------

\begin{abstract}
	\begin{justify}
		 \emph{Se logro el punto extra !}, en esta práctica se logro hacer un sumador de bits basico con capacidad de "carry in" y con un "carry out". 
		\end{justify}
{\it Keywords:}   diseño de hardware, sumador, bits
\end{abstract}
\begin{multicols}{2}
\section{Objetivo}\label{Objetivo}
El estudiante pone en práctica el procedimiento de diseño estructurado y 
jerárquico de circuitos combinacionales para generar un sumador de cuatro bits.
\section{Introducción}\label{sec:intro}
Los sumadores de números binarios nos sirven para que distintos circuitos puedan cumplir con ciertas operaciones aritméticas, y aunque hacer una suma binaria en papel es de las operaciones más sencillas que se pueden realizar, al pasar esta teoría a circuitos combinacionales todo se complica, ya que para un simple suma de dos números de un bit necesitamos de un sumador que a su vez esta conformado por dos medio sumadores y que estos deben estar conectados, por lo que al hacer una suma de números con mas bit, se necesitan la mimas cantidad de sumadores que de bits de los números a sumar, por lo que se vuelve una tarea complicada.
\section{Análisis Teórico}\label{sec:analiTeorico}
Un sumador de 4 bits es un circuito lógico combinacional que se utiliza para sumar dos números binarios de 4 bits. El sumador de 4 bits consta de cuatro sumadores completos de un bit (también conocidos como sumadores de medio bit) y un sumador completo de cuatro bits.

Cada sumador completo de un bit tiene dos entradas (A y B), una entrada de acarreo (Cin), dos salidas (S y Cout) y una tabla de verdad asociada. La tabla de verdad describe cómo se suman A, B y Cin para dar como resultado S y Cout. La salida S es el resultado de la suma de A, B y Cin, mientras que la salida Cout es el acarreo generado por la suma. La salida Cout de un sumador completo de un bit se conecta a la entrada Cin del siguiente sumador completo de un bit para generar el acarreo de la suma de los bits más significativos.

El sumador completo de cuatro bits suma los cuatro bits menos significativos de los números de entrada, utilizando los cuatro sumadores completos de un bit, y luego suma los acarreos de los sumadores completos de un bit para obtener el resultado final de la suma de los dos números de entrada.

El análisis teórico del sumador de 4 bits implica la identificación de la tabla de verdad del circuito, así como la determinación de los tiempos de propagación y los tiempos de retardo de cada sumador completo de un bit. El tiempo de propagación es el tiempo que tarda la señal de entrada en propagarse a través del circuito y generar una salida, mientras que el tiempo de retardo es el tiempo que tarda la señal de entrada en propagarse a través del circuito y generar una salida, más el tiempo que tarda la salida en estabilizarse.

También es importante tener en cuenta la capacidad de carga del circuito, que se refiere a la cantidad máxima de corriente que puede suministrarse o consumirse en cada salida del circuito sin afectar su funcionamiento. La capacidad de carga se puede calcular a partir de la resistencia de cada salida y la tensión de alimentación del circuito.
\end{multicols}
\section{Resultados esperados}\label{sec:resEsperados}
\begin{table}[H]
\centering	
\begin{tabular}{|c|c|c|c|c|c|c|c|c|c|c|c|c|c|}
\hline
$A_3$ & $A_2$ & $A_1$ & $A_0$ & $B_3$ & $B_2$ & $B_1$ & $B_0$ & $C_{in}$ & $S_3$ & $S_2$ & $S_1$ & $S_0$ & $C_{out}$\\
\hline
0 & 0 & 0 & 0 & 0 & 0 & 0 & 0 & 0 & 0 & 0 & 0 & 0 & 0\\
\hline
0 & 0 & 0 & 0 & 0 & 0 & 0 & 1 & 0 & 0 & 0 & 0 & 1 & 0\\
\hline
0 & 0 & 0 & 0 & 0 & 0 & 1 & 0 & 0 & 0 & 0 & 1 & 0 & 0\\
\hline
0 & 0 & 0 & 0 & 0 & 0 & 1 & 1 & 0 & 0 & 0 & 1 & 1 & 0\\
\hline
0 & 0 & 0 & 0 & 0 & 1 & 0 & 0 & 0 & 0 & 1 & 0 & 0 & 0\\
\hline
0 & 0 & 0 & 0 & 0 & 1 & 0 & 1 & 0 & 0 & 1 & 0 & 1 & 0\\
\hline
0 & 0 & 0 & 0 & 0 & 1 & 1 & 0 & 0 & 0 & 1 & 1 & 0 & 0\\
\hline
... & ... & ... & ... & ... & ... & ... & ... & ... & ... & ... & ... & ... & ... \\
\hline
\end{tabular}
\caption{Fragmento de tabla de verdad de todos los outputs del sumador de 4 bits}
\label{table:1}
\end{table}
\begin{multicols}{2}
	
\section{Resultados obtenidos}\label{sec:resObtenidos}
Prueba de que si obtuvimos los resultados esperados.
\begin{figure}[H]
	\centering	
	\includegraphics[scale=0.1]{../../static/carryOut.jpeg}
	\caption{Carry Out ejempo}
	\label{fig:1}
\end{figure}

\begin{figure}[H]
	\centering
	\includegraphics[scale=0.1]{../../static/repreSumaGrande.jpeg}
	\caption{Suma grande}
	\label{fig:2}
\end{figure}

\begin{figure}[H]
	\centering	
	\includegraphics[scale=0.1]{../../static/sample3.jpeg}	
	\caption{evidencia ejemplo 3}
	\label{fig:3}
\end{figure}

\begin{figure}[H]
	\centering	
	\includegraphics[scale=0.1]{../../static/unoIgualAUno.jpeg}
	\caption{Uno es igual a uno}
	\label{fig:4}
\end{figure}

\section{Conclusiones}\label{sec:conclusion}
Para este circuito combinacional se necesito de otros circuitos separados que fusionados y repitiendo esto 4 veces que es la cantidad de bits de los números a sumar, se pudo conseguir este sumador de 4 bits que es mucho más complejo que los circuitos implementados con en las practicas anteriores, ahora no solo había señales de entrada y salida, si no que ahora las señales de salida se convertían en señales de entrada que cumplían con ser implementadas otra vez para modificar la siguiente señal de salida, además de que si esto se hiciera con un display de 7 segmentos, estas salidas también tendrían que ser convertida a variables, para que le display pueda cambiar dependiendo del valor de las señales de salida.
\section*{Referencias}\label{sec:referencias}	
\begin{thebibliography}{1}						% -------------------- Bibliografía
\bibitem{mano2004digital}
Mano, M. M. (2004). Digital design (4th ed.). Upper Saddle River, NJ: Prentice Hall.
\end{thebibliography}
\end{multicols}
\end{document}

