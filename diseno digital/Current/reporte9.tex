\documentclass{article}
%\usepackage[spanish,activeacute]{babel}
%\usepackage[english,activeacute]{babel}
%\usepackage[latin1]{inputenc}
\usepackage[utf8]{inputenc}
\usepackage[english]{babel}
\usepackage{amsmath,amsfonts,amssymb,amstext,amsthm,amscd}
\usepackage{hyperref}
\usepackage{latexsym}
\usepackage{graphicx}
%\usepackage{subfigure}
\usepackage{subfig}
%\linespread{1.6}
\usepackage{float}
\usepackage{dcolumn}% Align table columns on decimal point(esto lo saque del ejemplo de revtex4)
\usepackage{bm}% bold math(esto lo saque del ejemplo de revtex4)
\newcounter{itemR}
\usepackage{here} \usepackage{fancyhdr}
%\usepackage{sidecap}
%\usepackage[spanish,activeacute]{babel}
\usepackage{multirow}
\usepackage{multicol}
\usepackage{array}
\usepackage{ragged2e}
%\usepackage{booktabs}% para hacer tablas profesionales con \toprule

% ------------------------------------------------------------------------------------------------------------------------------------------------------

\usepackage{fancyhdr}
\setlength{\headheight}{15.2pt}
\usepackage[paperwidth=8.5in, paperheight=11.0in, top=1.0in, bottom=1.0in, left=1.0in, right=1.0in]{geometry}

\pagestyle{fancyplain}
\fancyhead[LE,RO]{Reporte $\#$9}
\fancyhead[CE,CO]{}
\fancyhead[RE,LO]{P23-LRT2022-3}
\fancyfoot[LE,RO]{\thepage}
\fancyfoot[CE,CO]{Diseño digital, UDLAP}
\fancyfoot[RE,LO]{}

% ------------------------------------------------------------------------------------------------------------------------------------------------------
% ------------------------------------------------------------------------------------------------------------------------------------------------------
% ------------------------------------------------------------------------------------------------------------------------------------------------------

\begin{document}

\fancypagestyle{plain}{
   	\renewcommand{\headrulewidth}{1pt}
   	\renewcommand{\footrulewidth}{1pt}
}

\renewcommand{\footrulewidth}{1pt}
\renewcommand{\tablename}{Tabla}
\renewcommand{\figurename}{Figura}

% ------------------------------------------------------------------------------------------------------------------------------------------------------
% ------------------------------------------------------------------------------------------------------------------------------------------------------
% ------------------------------------------------------------------------------------------------------------------------------------------------------

\title{Circuitos digitales básicos}
\author{\small{Erick Gonzalez Parada ID: 178145 $\&$ Omar Martínez López ID: 177465}\\ 
\small{ Depto. Computación Electrónica y Mecatrónica.} \\
\small {Docente: Dr. Juan Carlos Moreno Rodríguez}}

\date{\small{\today}}

\maketitle

% ------------------------------------------------------------------------------------------------------------------------------------------------------

\begin{abstract}
	\begin{justify}
		Se realizo un demux en donde recibe como entrada los tres selectores que para la tarjeta de desarrollo son los switches que uno asigna en el pin planner, de ahí se configuro otro input
		el cual es un botón para mostrar el resultado que ejerce el demux donde la salida tiene que mostrar el número correspondiente con respecto a los switches. 
		\end{justify}
{\it Keywords:}   demultiplexor, switches 
\end{abstract}
\begin{multicols}{2}
\section{Objetivo}\label{Objetivo}
El estudiante pone en práctica el procedimiento de diseño para solucionar un 
problema empleando un circuito combinacional.
\section{Introducción}\label{sec:intro}
En esta practica se llevo a cabo el uso de un demultiplexor, ya que con ayuda de este se intento solucionar un problema propuesto, haciendo uso de las propiedades que este tiene, que son que puede mostrar el valor de una estrada, pero especificando en cual de las salidas se desea que se muestre esa entrada, gracias a unas variables de selección, todo esto implementado en un código de VHDL para una tarjeta DE10-Lite, ya que gracias a esta y representado un botón como la señal de entrada, switches como las variables de selección  y por ultimo leds como las señales de salida que mostraran el valor de la señal de entrada. Al configurar de cierta manera las switches de selección, se ajustará a un led especifico, el cual estará mostrando el valor de entrada del botón. 
\section{Análisis Teórico}\label{sec:analiTeorico}

Un demultiplexor binario, también conocido como demux binario, es un circuito lógico que tiene una entrada y dos o más salidas. En función del valor de los bits de la entrada, el demux dirige la señal de entrada a una de las salidas correspondientes.

Un demux binario de dos salidas tiene una entrada de un bit y dos salidas. Si el bit de entrada es 0, la señal de entrada se dirige a la primera salida, y si es 1, la señal de entrada se dirige a la segunda salida. El demux binario de dos salidas se puede representar mediante la siguiente tabla de verdad:

\begin{table}[H]
	\centering	
	\begin{tabular}{|c|c|c|}
	\hline
	Entrada & Salida 1 & Salida 2 \\
	\hline
	0 & 1 & 0 \\
	\hline
	1 & 0 & 1 \\
	\hline
	\end{tabular}
\end{table}

Un demux binario de $N$ salidas tiene una entrada de $\log_2(N)$ bits y $N$ salidas. El número de bits de entrada se calcula como el logaritmo en base 2 del número de salidas. Por ejemplo, un demux binario de cuatro salidas tiene dos bits de entrada. La tabla de verdad para un demux binario de cuatro salidas se muestra a continuación:

\begin{table}[H]
	\centering	
	\begin{tabular}{|c|c|c|c|c|c|}
	\hline
	Entrada 1 & Entrada 0 & Salida 0 & Salida 1 & Salida 2 & Salida 3 \\
	\hline
	0 & 0 & 1 & 0 & 0 & 0 \\
	\hline
	0 & 1 & 0 & 1 & 0 & 0 \\
	\hline
	1 & 0 & 0 & 0 & 1 & 0 \\
	\hline
	1 & 1 & 0 & 0 & 0 & 1 \\
	\hline
	\end{tabular}

\end{table}	
En resumen, un demux binario es un circuito lógico que dirige la señal de entrada a una de las salidas correspondientes en función del valor de los bits de la entrada. La cantidad de bits de entrada y salidas depende del número de salidas requerido.

\end{multicols}
\section{Resultados esperados}\label{sec:resEsperados}
Estos son los resultados esperados ver tabla \ref{tab:1}.

\begin{table}[H]
	\centering
	\begin{tabular}{|c|c|c||c|c|c|c|c|c|c|c|c|c|c|c|}
\hline
$A$ & $B$ & $C$ & $Y_0$ & $Y_1$ & $Y_2$ & $Y_3$ & $Y_4$ & $Y_5$ & $Y_6$ & $Y_7$ \\
\hline
0 & 0 & 0 & 1 & 0 & 0 & 0 & 0 & 0 & 0 & 0 \\
\hline
0 & 0 & 1 & 0 & 1 & 0 & 0 & 0 & 0 & 0 & 0 \\
\hline
0 & 1 & 0 & 0 & 0 & 1 & 0 & 0 & 0 & 0 & 0 \\
\hline
0 & 1 & 1 & 0 & 0 & 0 & 1 & 0 & 0 & 0 & 0 \\
\hline
1 & 0 & 0 & 0 & 0 & 0 & 0 & 1 & 0 & 0 & 0 \\
\hline
1 & 0 & 1 & 0 & 0 & 0 & 0 & 0 & 1 & 0 & 0 \\
\hline
1 & 1 & 0 & 0 & 0 & 0 & 0 & 0 & 0 & 1 & 0 \\
\hline
1 & 1 & 1 & 0 & 0 & 0 & 0 & 0 & 0 & 0 & 1 \\
\hline
\end{tabular}	
\caption{resultados esperados}
\label{tab:1}
\end{table}
\section{Resultados obtenidos}\label{sec:resObtenidos}

Estos fueron los resultados obtenidos, como se puede observar se cumplió el objetivo.

\begin{figure}[H]
	\centering
	\includegraphics[scale=0.3]{../../static/cerosO.jpeg}	
	\caption{ceros ocultos}
	\label{fig:1}	
\end{figure}

\begin{figure}[H]
	\centering
	\includegraphics[scale=0.3]{../../static/ceros.jpeg}	
	\caption{ceros}
	\label{fig:2}	
\end{figure}

\begin{figure}[H]
	\centering
	\includegraphics[scale=0.3]{../../static/cincosO.jpeg}	
	\caption{cincos ocultos}
	\label{fig:3}	
\end{figure}

\begin{figure}[H]
	\centering
	\includegraphics[scale=0.3]{../../static/cincos.jpeg}	
	\caption{cincos}
	\label{fig:4}	
\end{figure}

\begin{figure}[H]
	\centering
	\includegraphics[scale=0.3]{../../static/sietesO.jpeg}	
	\caption{sietesO ocultos}
	\label{fig:5}	
\end{figure}

\begin{figure}[H]
	\centering
	\includegraphics[scale=0.3]{../../static/sietes.jpeg}	
	\caption{sietes}
	\label{fig:6}	
\end{figure}

\begin{multicols}{2}
\section{Conclusiones}\label{sec:conclusion}
Esta practica fue sencilla ya que únicamente se llevo a cabo un ejemplo de un uso de un demultiplexor, pero también muy útil ya que esto nos abre muchas posibilidades a solucionar posibles problemas futuros con el uso de estas herramienta, y así como podríamos utilizar este demultiplexor, también podríamos trabajar con multiplexores que tienen un funcionamiento parecido pero al mismo tiempo sirve para hacer lo contrario, o con codificadores y decodificadores que también tienen muchos usos para la solución de problemas. Al final se nota que es útil el poder elegir cual es la salida con la que vamos a hacer interactuar nuestra entrada.

\begin{thebibliography}{1}						% -------------------- Bibliografía
\bibitem{mano2004digital}
Mano, M. M. (2004). Digital design (4th ed.). Upper Saddle River, NJ: Prentice Hall.
\end{thebibliography}
\end{multicols}
\end{document}

