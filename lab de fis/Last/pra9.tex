\documentclass{article}
%\usepackage[spanish,activeacute]{babel}
%\usepackage[english,activeacute]{babel}
%\usepackage[latin1]{inputenc}
\usepackage[utf8]{inputenc}
\usepackage[english]{babel}

\usepackage{amsmath,amsfonts,amssymb,amstext,amsthm,amscd}
\usepackage{hyperref}
\usepackage{latexsym}
\usepackage{graphicx}
%\usepackage{subfigure}
\usepackage{subfig}
%\linespread{1.6}
\usepackage{float}
\usepackage{dcolumn}% Align table columns on decimal point(esto lo saque del ejemplo de revtex4)
\usepackage{bm}% bold math(esto lo saque del ejemplo de revtex4)
\newcounter{itemR}
\usepackage{here} %recordar usar el comando[H] para las gráficas que es el comando here en lugar de [h!]
\usepackage{fancyhdr}
%\usepackage{sidecap}
%\usepackage[spanish,activeacute]{babel}
\usepackage{multirow}
\usepackage{multicol}
\usepackage{array}
\usepackage{enumitem}
%\usepackage{booktabs}% para hacer tablas profesionales con \toprule

% ------------------------------------------------------------------------------------------------------------------------------------------------------

\usepackage{fancyhdr}
\setlength{\headheight}{15.2pt}
\usepackage[paperwidth=8.5in, paperheight=11.0in, top=1.0in, bottom=1.0in, left=1.0in, right=1.0in]{geometry}

\pagestyle{fancyplain}
\fancyhead[LE,RO]{Práctica $\#$9, colisiones}
\fancyhead[CE,CO]{}
\fancyhead[RE,LO]{P23-FIS1012-12}
\fancyfoot[LE,RO]{\thepage}
\fancyfoot[CE,CO]{Laboratorio de Física, UDLAP}
\fancyfoot[RE,LO]{}

% ------------------------------------------------------------------------------------------------------------------------------------------------------
% ------------------------------------------------------------------------------------------------------------------------------------------------------
% ------------------------------------------------------------------------------------------------------------------------------------------------------

\begin{document}

\fancypagestyle{plain}{
   	\renewcommand{\headrulewidth}{1pt}
   	\renewcommand{\footrulewidth}{1pt}
}

\renewcommand{\footrulewidth}{1pt}
\renewcommand{\tablename}{Tabla}
\renewcommand{\figurename}{Figura}

% ------------------------------------------------------------------------------------------------------------------------------------------------------
% ------------------------------------------------------------------------------------------------------------------------------------------------------
% ------------------------------------------------------------------------------------------------------------------------------------------------------

\title{Colisiones elásticas}
\author{\small{Luis Alberto Gil Bocanegra ID: 177410, Erick Gonzalez Parada ID: 178145}\\
 \small{Daniela Hernández García ID: 179051 $\&$ Luis Francisco Avila Romero ID: 177632.}\\		% ----- Varios autores separarlos por comas:  \small{Nombre(s) de (los) autor(es)\footnote{ID; correo@udlap.mx}, Nombre(s) de (los) autor(es)\footnote{ID; correo@udlap.mx}
	   \small{Depto. de Actuaría, Física y Matemáticas, Universidad de las Américas Puebla, Puebla, M\'exico 72810}}
\date{\small{\today}}

\maketitle

% ------------------------------------------------------------------------------------------------------------------------------------------------------
% ------------------------------------------------------------------------------------------------------------------------------------------------------
% ------------------------------------------------------------------------------------------------------------------------------------------------------

\begin{abstract}
Esta práctica se basó en demostrar que el momento lineal y la energía cinética neta se conserva durante
la colisión elástica frontal de dos objetos en un sistema con fricción despreciable, mediante dos
móviles en un riel rectangular impulsados por una fuerza y tomando las velocidades de estos al momento
y después del choque, donde al final se pudo observar que el objeto que inicia con mayor velocidad antes de colisionar es el mismo que termina con mayor velocidad después de la colisión para el
caso de las mismas masas, de masas diferentes , se compensa con la velocidad a la que este colisiona con el móvil 2.
\\
\\
{\it Keywords:}  colisión, energía  
\\
\\
\end{abstract}

% ------------------------------------------------------------------------------------------------------------------------------------------------------

\begin{multicols}{2}

\section{Desarrollo teórico}\label{Desarrollo Teorico}                              	% -------------------- Introducción
Una colisión elástica es aquella en la que no hay disipación de energía, sino que se conserva la cantidad de movimiento total antes y después de la colisión. Durante una colisión elástica, las partículas se comportan como bolas duras que se rebotan unas contra otras.

\subsection{Conservación de la cantidad de movimiento}

La cantidad de movimiento total del sistema antes y después de una colisión elástica es la misma, ya que no hay fuerzas externas actuando. Si denominamos $\mathbf{p}{1}$ y $\mathbf{p}{2}$ a las cantidades de movimiento de las dos partículas antes de colisionar y $\mathbf{p}'{1}$ y $\mathbf{p}'{2}$ a las cantidades de movimiento después de colisionar, la conservación de la cantidad de movimiento se expresa como:

\begin{equation}
\mathbf{p}{1} + \mathbf{p}{2} = \mathbf{p}'{1} + \mathbf{p}'{2}
\end{equation}

\subsection{Conservación de la energía cinética}

Similarmente, la energía cinética total es también conservada en una colisión elástica. Si las energías cinéticas de las partículas antes de colisionar son $E_{c1}$ y $E_{c2}$, y después de colisionar son $E'{c1}$ y $E'{c2}$, se cumple que:

\begin{equation}
E_{c1} + E_{c2} = E'{c1} + E'{c2}
\end{equation}

De las ecuaciones anteriores se deduce que las colisiones elásticas implican un intercambio de cantidad de movimiento y energía cinética entre las partículas, pero la cantidad total se conserva. Esto da lugar a rebotes elásticos en lugar de fusiones.


\section{Desarrollo Experimental}\label{Desarrollo experimental}				% -------------------- Metodología 

\begin{figure}[H]
	\centering	
	\includegraphics[scale=0.4]{../../static/fig1d9.png}
	\caption{Montaje del riel con ambos móviles y materiales. Representación del choque.}
	\label{fig:1}
\end{figure}

Calibración del sistema

\begin{itemize}
\item Colocar el riel triangular sobre la mesa y alinear horizontalmente. Conectar la
manguera del compresor al riel para introducir aire y de esta forma reducir la fricción.
\item Pegar con cinta las regletas sobre los móviles, estas servirán de bandera para
detección de las velocidades en las fotocompuertas.
\item Colocar las fotocompuertas a distancias tales que puedan medir la velocidad del
móvil antes y después de la colisión (fotocompuerta 1 que mida la velocidad del
móvil 1 y la fotocompuerta 2 mida la velocidad del móvil 2).
\item Colocar a los móviles el bloque con unas ligas para evitar que no choquen
directamente. (Como se muestra en fig1) 
\item En el Smart Timer seleccionar Speed —> Collision
\end{itemize}

Procedimiento experimental

\begin{itemize}
\item Lanzar ambos bloques tal que la colisión se lleve en el espacio de las
fotocompuertas y detectar medición de ambas velocidades (antes y después del
choque) para ambos bloques.
\item Repetir para diferentes configuraciones, al cambiar la velocidad de los bloques (la
fuerza de lanzamiento) y las masas.
\end{itemize}

Consideraciones generales

\begin{itemize}
\item Asegurarse que los móviles no choquen directamente uno con el otro o en los
extremos del riel.
\item No aplicar demasiada fuerza para evitar que los bloques salgan del riel.
\item Asegurarse que las regletas estén bien puestas y que las fotocompuertas estén a
una adecuada altura para detectarlas 
\item Asegurarse que el Smart Timer tome dos velocidades de cada móvil.
\item Observar que los móviles choquen en el espacio entre ambas fotocompuertas para
obtener la medición.
\end{itemize}


\end{multicols}
\section{Resultados y análisis}\label{Resultados}			% -------------------- Resultados

Se realizaron 3 mediciones por caso, dichas mediciones se ven representadas en las tablas \ref{tab:1} y \ref{tab:2} que se tienen en la parte de abajo del documento, en las cuales hay 2 tablas en las que los móviles tienen la misma cantidad de masa, mientras que las otras 2, las masas son diferentes. Primero el enfoque irá dirigido a cuando las masas de los móviles son iguales, se puede apreciar que, en el primer intento, el móvil 1 de 184g $\pm$ 0.5 sale con una velocidad de 50.7 cm/s $\pm$ 0.1 mientras que el móvil 2 de igual magnitud de masa sale con una velocidad de 33.2 cm/s $\pm$ 0.1, colisionan y las velocidades de estos cambia considerablemente, siendo de 11.3cm/s la del móvil 1 mientras que la del móvil 2 es de 10.7 cm/s $\pm$ 0.1. En el segundo lanzamiento, el móvil 1 de 184g $\pm$ 0.5 sale con una velocidad de 38.1 cm/s $\pm$ 0.1 mientras que el móvil 2 de igual magnitud de masa sale con una velocidad de 26.1 cm/s $\pm$ 0.1, colisionan y las velocidades de estos cambia considerablemente, siendo de 7.3 cm/s $\pm$ 0.1 la del móvil 1 mientras que la del móvil 2 es de 0.9 cm/s $\pm$ 0.1. Para el último lanzamiento, el móvil 1 de 184g sale con una velocidad de 105.2 cm/s mientras que el móvil 2 de igual magnitud de masa sale con una velocidad de 0.1 cm/s $\pm$ 0.1, colisionan y las velocidades de estos cambia considerablemente, siendo de 12.3 cm/s $\pm$ 0.1 la del móvil 1 mientras que la del móvil 2 es de 2.7 cm/s $\pm$ 0.1. Al observar las mediciones realizadas, se hace ver que, normalmente, en este caso en que las masas son iguales, el objeto que inicia con mayor velocidad antes de colisionar es el mismo que termina con mayor velocidad después de la colisión.

\begin{table}[H]
	\centering	
	\begin{tabular}{|c|c|c|c|}
\hline
M1 $\pm$ 0.5 (g) & V1 $\pm$ 0.1 (m/s) & M2 $\pm$ 0.5 (g) & V2 $\pm$ 0.1 (m/s) \\
\hline
184.0 & 50.7 & 184.0 & 33.2 \\
\hline
184.0 & 38.1 & 184.0 & 26.1 \\
\hline
184.0 & 105.2 & 184.0 & 0.1 \\
\hline
\end{tabular}
\caption{Antes de la colisión}
\label{tab:1}
\end{table}

\begin{table}[H]
	\centering	
	\begin{tabular}{|c|c|c|c|}
\hline
M1 $\pm$ 0.5 (g) & V1 $\pm$ 0.1 (m/s) & M2 $\pm$ 0.5 (g) & V2 $\pm$ 0.1 (m/s) \\
\hline
184.0 & 11.3 & 184.0 & 10.7 \\
\hline
184.0 & 7.3 & 184.0 & 0.9 \\
\hline
184.0 & 12.3 & 184.0 & 2.7 \\
\hline
\end{tabular}
\caption{Después de la colisión}
\label{tab:2}
\end{table}

Ahora va el caso en el que los móviles poseen masas diferentes tablas \ref{tab:3} y \ref{tab:4}, el móvil 1 posee una masa de 184g $\pm$ 0.5 mientras que el móvil 2 posee una masa de 218g $\pm$ 0.5. En el primer lanzamiento, el móvil 1 de 184g $\pm$ 0.5 sale con una velocidad de 95.2 cm/s $\pm$ 0.1 mientras que el móvil 2 de 218g $\pm$ 0.5 de masa, sale con una velocidad de 48.0 cm/s $\pm$ 0.1, colisionan y las velocidades de dichos móviles cambia, siendo de 14.1 cm/s $\pm$ 0.1 la velocidad con la que resulta en móvil 1, mientras que el móvil 2 resulta ser de 13.3 cm/s $\pm$ 0.1. En el segundo lanzamiento, el móvil 1 de 184g $\pm$ 0.5 sale con una velocidad de 120.4 cm/s $\pm$ 0.1 mientras que el móvil 2 de 218g $\pm$ 0.5 de masa, sale con una velocidad de 39.3 cm/s $\pm$ 0.1, colisionan y las velocidades de dichos móviles cambia, siendo de 13.0 cm/s $\pm$ 0.1 la velocidad con la que resulta en móvil 1, mientras que el móvil 2 resulta ser de 12.2 cm/s $\pm$ 0.1. Por último, en el tercer lanzamiento el móvil 1 de 184g $\pm$ 0.5 sale con una velocidad de 59.1 cm/s $\pm$ 0.1 mientras que el móvil 2 de 218g $\pm$ 0.5 de masa, sale con una velocidad de 59.8 cm/s $\pm$ 0.1, colisionan y las velocidades de dichos móviles cambia, siendo de 13.5 cm/s $\pm$ 0.1 la velocidad con la que resulta en móvil 1, mientras que el móvil 2 resulta ser de 13.8 cm/s $\pm$ 0.1. Al observar las mediciones, se puede ver que, a pesar de que el móvil 2 tenía más masa que el móvil 1, el móvil 2 iba bastante más lento que el móvil 1, esta condición resulta en que la mayoría de las velocidades finales sean similares, ya que la falta de masa que hay en el móvil 1, se compensa con la velocidad a la que este colisiona con el móvil 2.

\begin{table}[H]
	\centering	
	\begin{tabular}{|c|c|c|c|}
\hline
M1 $\pm$ 0.5 (g) & V1 $\pm$ 0.1 (m/s) & M2 $\pm$ 0.5 (g) & V2 $\pm$ 0.1 (m/s) \\
\hline
184.0 & 95.2 & 218.0 & 48.0 \\
\hline
184.0 & 120.4 & 218.0 & 39.3 \\
\hline
184.0 & 59.1 & 218.0 & 59.8 \\
\hline
\end{tabular}
\caption{Antes de la colisión}
\label{tab:3}
\end{table}

\begin{table}[H]
	\centering	
	\begin{tabular}{|c|c|c|c|}
\hline
M1 $\pm$ 0.5 (g) & V1 $\pm$ 0.1 (m/s) & M2 $\pm$ 0.5 (g) & V2 $\pm$ 0.1 (m/s) \\
\hline
184.0 & 14.1 & 218.0 & 13.3 \\
\hline
184.0 & 13.0 & 218.0 & 12.2 \\
\hline
184.0 & 13.5 & 218.0 & 13.8 \\
\hline
\end{tabular}
	\caption{Después de la colisión}
	\label{tab:4}
\end{table}

Al calcular la conservación, tanto de momento como de cinética, se puede apreciar que hay una pérdida significativa en ambas situaciones y en cada uno de los casos medidos, esto se debe a una pérdida de momentos por calor al momento del choque entre ambos móviles, ya que al impactar ambos con velocidades diferentes, se crea cierta fricción en un cuestión de microsegundos, esta fricción es suficiente para hacer que exista una pérdida considerable de momento y de cinética, dando como resultado los valores calculados en las tablas \ref{tab:5}, \ref{tab:6}, \ref{tab:7} y \ref{tab:8}.

\begin{table}[H]
	\centering	
	\begin{tabular}{|c|c|}
\hline
ANTES $\pm$ 0.5 & DESPUÉS $\pm$ 0.5 \\
\hline
1543.7 kg$\cdot$m/s & 4048.0 kg$\cdot$m/s \\
\hline
1181.2 kg$\cdot$m/s & 1508.8 kg$\cdot$m/s \\
\hline
1937.5 kg$\cdot$m/s & 2760.0 kg$\cdot$m/s \\
\hline
\end{tabular}
	\caption{Conservación del momento lineal con masas iguales}
	\label{tab:5}
\end{table}

\begin{table}[H]
	\centering	
	\begin{tabular}{|c|c|}
\hline
ANTES $\pm$ 0.5 & DESPUÉS $\pm$ 0.5 \\
\hline
3378.9 kg$\cdot$m/s & 2258.1 kg$\cdot$m/s \\
\hline
1962.1 kg$\cdot$m/s & 4977.2 kg$\cdot$m/s \\
\hline
1937.5 kg$\cdot$m/s & 1458.9 kg$\cdot$m/s \\
\hline
\end{tabular}	
	\caption{Conservación de energía cinética con masas iguales}
	\label{tab:6}
\end{table}

\begin{table}[H]
	\centering	
	\begin{tabular}{|c|c|}
\hline
ANTES $\pm$ 0.5 & DESPUÉS $\pm$ 0.5 \\
\hline
2798.1 kg$\cdot$m/s & 5493.8 kg$\cdot$m/s \\
\hline
3072.1 kg$\cdot$m/s & 5051.6 kg$\cdot$m/s \\
\hline
2391.1 kg$\cdot$m/s & 5492.4 kg$\cdot$m/s \\
\hline
\end{tabular}
	\caption{Conservación del momento con masas diferentes}
	\label{tab:7}
\end{table}

\begin{table}[H]
	\centering	
	\begin{tabular}{|c|c|}
\hline
ANTES $\pm$ 0.5 & DESPUÉS $\pm$ 0.5 \\
\hline
10849.5 kg$\cdot$m/s & 37571.5 kg$\cdot$m/s \\
\hline
15019.9 kg$\cdot$m/s & 31771.5 kg$\cdot$m/s \\
\hline
71112.6 kg$\cdot$m/s & 37524.9 kg$\cdot$m/s \\
\hline
\end{tabular}
	\caption{Conservación de energía cinética con masas diferentes}
	\label{tab:8}
\end{table}

\section{Conclusiones}\label{Conclusiones}				% -------------------- Conclusiones
Técnicamente cumpliendo el objetivo que fue calcular el trabajo realizado del objeto volvemos a subrayar que se observa que el trabajo teórico es mayor que el trabajo experimental debido a que las magnitudes utilizadas en el trabajo teórico son constantes, mientras que en el trabajo experimental no lo son debido a diversos fenómenos que pueden afectar el resultado final. Esto sugiere que la precisión y la fiabilidad del trabajo teórico es mayor que la del trabajo experimental debido a la capacidad de controlar y predecir las magnitudes utilizadas. Sin embargo, se debe tener en cuenta que el trabajo experimental puede proporcionar información valiosa sobre cómo los diversos factores pueden afectar el resultado final y, por lo tanto, debe realizarse con precaución y considerando todas las variables relevantes.

\begin{thebibliography}{9}						% -------------------- Bibliografía
	\bibitem{Martín}
		Martín, I. (2004). Física General
		\bibitem{Serway}
		Serway, R. A., $\&$ Jewett, J. W. (2008). Física para ciencias e ingeniería. (7.a
ed., Vol. 1). CENGAGE Learning.

\bibitem{Pérez}
	Newton, I. (1687). Philosophiæ Naturalis Principia Mathematica [Mathematical Principles of Natural Philosophy]. Londini: Jussu Societatis Regiæ ac Typis Josephi Streater.
\end{thebibliography}

\end{document}	