\documentclass{article}
%\usepackage[spanish,activeacute]{babel}
%\usepackage[english,activeacute]{babel}
%\usepackage[latin1]{inputenc}
\usepackage[utf8]{inputenc}
\usepackage[english]{babel}

\usepackage{amsmath,amsfonts,amssymb,amstext,amsthm,amscd}
\usepackage{hyperref}
\usepackage{latexsym}
\usepackage{graphicx}
%\usepackage{subfigure}
\usepackage{subfig}
%\linespread{1.6}
\usepackage{float}
\usepackage{dcolumn}% Align table columns on decimal point(esto lo saque del ejemplo de revtex4)
\usepackage{bm}% bold math(esto lo saque del ejemplo de revtex4)
\newcounter{itemR}
\usepackage{here} %recordar usar el comandao[H] Para las  graficas que es el comando here en lugar de [h!]
\usepackage{fancyhdr}
%\usepackage{sidecap}
%\usepackage[spanish,activeacute]{babel}
%\usepackage{multirow}
\usepackage{multicol}
%\usepackage{array}
%\usepackage{booktabs}% para hacer tablas profesionales con \toprule

% ------------------------------------------------------------------------------------------------------------------------------------------------------

\usepackage{fancyhdr}
\setlength{\headheight}{15.2pt}
\usepackage[paperwidth=8.5in, paperheight=11.0in, top=1.0in, bottom=1.0in, left=1.0in, right=1.0in]{geometry}

\pagestyle{fancyplain}
\fancyhead[LE,RO]{Pr\'actica $\#$4, MRU}
\fancyhead[CE,CO]{}
\fancyhead[RE,LO]{P23-FIS1012-12}
\fancyfoot[LE,RO]{\thepage}
\fancyfoot[CE,CO]{Laboratorio de F\'isica, UDLAP}
\fancyfoot[RE,LO]{}

% ------------------------------------------------------------------------------------------------------------------------------------------------------
% ------------------------------------------------------------------------------------------------------------------------------------------------------
% ------------------------------------------------------------------------------------------------------------------------------------------------------

\begin{document}

\fancypagestyle{plain}{
   	\renewcommand{\headrulewidth}{1pt}
   	\renewcommand{\footrulewidth}{1pt}
}

\renewcommand{\footrulewidth}{1pt}
\renewcommand{\tablename}{Tabla}
\renewcommand{\figurename}{Figura}

% ------------------------------------------------------------------------------------------------------------------------------------------------------
% ------------------------------------------------------------------------------------------------------------------------------------------------------
% ------------------------------------------------------------------------------------------------------------------------------------------------------

\title{Movimiento Rectilíneo Uniforme}
\author{\small{Erick Gonzalez Parada ID: 178145, Leonardo Escamilla Salgado ID: 179021 $\&$ Daniela Lomán Barrueta ID: 179062.}\\		% ----- Varios autores separarlos por comas:  \small{Nombre(s) de (los) autor(es)\footnote{ID; correo@udlap.mx}, Nombre(s) de (los) autor(es)\footnote{ID; correo@udlap.mx}
	   \small{Depto. de Actuaría, Física y Matemáticas, Universidad de las Américas Puebla, Puebla, M\'exico 72810}}
\date{\small{\today}}

\maketitle

% ------------------------------------------------------------------------------------------------------------------------------------------------------
% ------------------------------------------------------------------------------------------------------------------------------------------------------
% ------------------------------------------------------------------------------------------------------------------------------------------------------

\begin{abstract}

El experimento consistio en montar el riel triangular con un móvil con minima fricción de tal manera que se pudo observar las características de movimiento rectilíneo uniforme medimos el tiempo que tardo el movil de un punto A a un punto B y lo coleccionamos.
\\
\\
{\it Keywords:}  Velocidad, Uniforme, tiempo
\\
\\
\end{abstract}

% ------------------------------------------------------------------------------------------------------------------------------------------------------

\begin{multicols}{2}



\section*{Desarrollo teórico}\label{Desarrollo Teorico}                              	% -------------------- Introduccion

El movimiento rectilíneo uniforme (MRU) es un tipo de movimiento en el que un objeto se mueve en línea recta a una velocidad constante. Es decir, la velocidad del objeto no varía en el tiempo y su trayectoria es una línea recta.
\\

El MRU es un concepto fundamental en la física, ya que permite entender y describir muchos otros movimientos más complejos. Además, se utiliza en muchas aplicaciones prácticas, como en la física de partículas, la ingeniería, la navegación y la ciencia espacial.
\\
\\
En términos matemáticos, la velocidad constante del objeto en MRU se expresa mediante la ecuación:
\begin{equation}\label{Ec:1}
	Vprom = d/t
\end{equation}
donde "v" es la velocidad del objeto en metros por segundo (m/s), "d" es la distancia recorrida en metros (m) y "t" es el tiempo transcurrido en segundos (s).
\\
También podemos expresar la posición del objeto en MRU mediante la ecuación:
\begin{equation}\label{Ec:2}
	x = xo + vt
\end{equation}
donde "x" es la posición del objeto en metros (m), "xo" es la posición inicial del objeto en metros (m) y "t" es el tiempo transcurrido en segundos (s).
\\
\\
Es importante destacar que, en MRU, la aceleración del objeto es nula, ya que su velocidad no cambia. La aceleración se define como el cambio en la velocidad del objeto por unidad de tiempo, y al ser constante, la aceleración en MRU es cero.
\\
\\
Importante mencionar que nuestro objetivo es obtener la relación de proporcionalidad para un móvil que se mueve con mínima fricción sobre un riel de aire horizontal. 
%bibliografía
\section*{Desarrollo Experimental}\label{Desarrollo experimental}				% -------------------- Metodolog'ia

Lo primero que se realizo fue analizar como quedarían nuestras tablas para poder trazar nuestras gráficas,\\
después de que eso quedase claro montamos el riel triangular que se encuentra debajo nuestra mesa y lo conectamos con el compresor para que la fricción de cualquier cosa que se monte en el riel sufra una fricción mínima.
Conectamos a la corriente el smart timer y el compresor de aire (para comprobar que el riel sacase aire). Las foto compuertas se atornillaron a una base con varas que permitían que estas llegasen a la altura para que puedan enviar la señal al pasar la bandera (pedazo de metal que se monta al riel para que los sensores de movimiento de las foto compuertas manden la señal).  

\begin{figure}[H]
	\centering
	\includegraphics[scale=2.5]{../static/skeleteonpra4.png}
	\caption{Diagrama de mesa de trabajo}
	\label{fig:1}
\end{figure}

Ya una vez que se monto todo se acomodaron unas ligas en los extremos del riel para que el móvil rebotara por lo tanto en ese punto empezamos a acomodar las foto compuertas de tal manera que nos dieran una distancia de 10 en 10 centímetros.
Con esa configuración empezamos a recolectar los tiempos que arrojaba el smart timer.  

\end{multicols}
\section*{Resultados y análisis}\label{Resultados}			% -------------------- Resultados

En la tabla \ref{Tabla:1} se muestran las 5 mediciones de tiempo obtenidas referentes a cada trayecto recorrido por el móvil con intervalos de 10 cm (en la tabla \ref{Tabla:1}, toda unidad en cm se paso a metros).
\begin{table}[H]
	\centering
	\begin{tabular}{|c|c|c|c|c|c|c|c|c|c|c|}
	\hline
	Distancia (m) & 0.1 & 0.2 & 0.3 & 0.4 & 0.5 & 0.6 & 0.7 & 0.8 & 0.9 & 1 \\
	\hline
	$t_1$ (s) $\pm$ 0.0001s & 0.2162 & 0.4001 & 0.5981 & 0.8284 & 0.9561 & 1.1616 & 1.4172 & 1.7819 & 1.9452 & 1.9523 \\
	\hline
	$t_2$ (s) $\pm$ 0.0001s & 0.2077 & 0.4243 & 0.6986 & 0.7967 & 0.9321 & 1.195 & 1.3454 & 1.9464 & 1.7747 & 1.9628 \\
	\hline
	$t_3$ (s) $\pm$ 0.0001s & 0.2037 & 0.4191 & 0.6595 & 0.769 & 1.0328 & 1.206 & 1.4833 & 1.4852 & 1.8828 & 2.224 \\
	\hline
	$t_4$ (s) $\pm$ 0.0001s & 0.2323 & 0.4161 & 0.7455 & 0.8104 & 0.9685 & 1.1826 & 1.4456 & 1.5394 & 1.7417 & 2.0288 \\
	\hline
	$t_5$ (s) $\pm$ 0.0001s & 0.2347 & 0.4342 & 0.6039 & 0.79 & 0.9785 & 1.1605 & 1.4761 & 1.6872 & 2.0531 & 1.5468 \\
	\hline
	\end{tabular}
	\caption{Tiempo medido con respecto a distancia}
	\label{Tabla:1}
\end{table}


%\begin{figure}[H]
%	\centering
%	\includegraphics[scale = 0.5]{fig2}
%	\caption{Posición de la partícula en función del tiempo donde los datos abarcan toda el área de la gráfica}
%	\label{Fig:1}
%\end{figure}



\begin{multicols}{2}
\section*{Conclusiones}\label{Conclusiones}				% -------------------- Conclusiones

En esta sección se deben exponer las conclusiones finales del trabajo sin enlistar. Las conclusiones consisten en al menos tres aspectos importantes: 1) Una comparación entre los resultados experimentales y la teoría: esta parte reponde a la preguntas ¿los resultados experimentales concuerdan con el modelo físico teórico? ¿cuánto concuerdan?, la comparación se escribe usando como referencia las secciones "Desarrollo teórico" y "Resultados y Análisis". 2) Una discusión sobre las posibles fuentes de error y una discusión sobre los valores numéricos de los errores calculados en la sección "Resultados y Análisis". 3) Propuestas necesarias para mejorar tanto el modelo teórico como la metodología experimental de la práctica.


\begin{thebibliography}{9}						% -------------------- Bibliograf'ia

%\bibitem{Einstein}
%   Albert Einstein,   \emph{The world as I see it}.   BN %Publishing,   2005.
%
\bibitem{Pérez}
	Pérez H, \emph{Física general 2021}
\end{thebibliography}

% ------------------------------------------------------------------------------------------------------------------------------------------------------
% ------------------------------------------------------------------------------------------------------------------------------------------------------
% ------------------------------------------------------------------------------------------------------------------------------------------------------

\end{multicols}

\end{document}										% -------------------- End Document