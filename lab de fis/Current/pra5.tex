\documentclass{article}
%\usepackage[spanish,activeacute]{babel}
%\usepackage[english,activeacute]{babel}
%\usepackage[latin1]{inputenc}
\usepackage[utf8]{inputenc}
\usepackage[english]{babel}

\usepackage{amsmath,amsfonts,amssymb,amstext,amsthm,amscd}
\usepackage{hyperref}
\usepackage{latexsym}
\usepackage{graphicx}
%\usepackage{subfigure}
\usepackage{subfig}
%\linespread{1.6}
\usepackage{float}
\usepackage{dcolumn}% Align table columns on decimal point(esto lo saque del ejemplo de revtex4)
\usepackage{bm}% bold math(esto lo saque del ejemplo de revtex4)
\newcounter{itemR}
\usepackage{here} %recordar usar el comandao[H] Para las  graficas que es el comando here en lugar de [h!]
\usepackage{fancyhdr}
%\usepackage{sidecap}
%\usepackage[spanish,activeacute]{babel}
\usepackage{multirow}
\usepackage{multicol}
%\usepackage{array}
%\usepackage{booktabs}% para hacer tablas profesionales con \toprule

% ------------------------------------------------------------------------------------------------------------------------------------------------------

\usepackage{fancyhdr}
\setlength{\headheight}{15.2pt}
\usepackage[paperwidth=8.5in, paperheight=11.0in, top=1.0in, bottom=1.0in, left=1.0in, right=1.0in]{geometry}

\pagestyle{fancyplain}
\fancyhead[LE,RO]{Práctica $\#$5, 2nda ley de Newton}
\fancyhead[CE,CO]{}
\fancyhead[RE,LO]{P23-FIS1012-12}
\fancyfoot[LE,RO]{\thepage}
\fancyfoot[CE,CO]{Laboratorio de Física, UDLAP}
\fancyfoot[RE,LO]{}

% ------------------------------------------------------------------------------------------------------------------------------------------------------
% ------------------------------------------------------------------------------------------------------------------------------------------------------
% ------------------------------------------------------------------------------------------------------------------------------------------------------

\begin{document}

\fancypagestyle{plain}{
   	\renewcommand{\headrulewidth}{1pt}
   	\renewcommand{\footrulewidth}{1pt}
}

\renewcommand{\footrulewidth}{1pt}
\renewcommand{\tablename}{Tabla}
\renewcommand{\figurename}{Figura}

% ------------------------------------------------------------------------------------------------------------------------------------------------------
% ------------------------------------------------------------------------------------------------------------------------------------------------------
% ------------------------------------------------------------------------------------------------------------------------------------------------------

\title{Segunda Ley de Newton y movimiento uniformemente acelerado}
\author{\small{Erick Gonzalez Parada ID: 178145, Leonardo Escamilla Salgado ID: 179021 $\&$ Daniela Lomán Barrueta ID: 179062.}\\		% ----- Varios autores separarlos por comas:  \small{Nombre(s) de (los) autor(es)\footnote{ID; correo@udlap.mx}, Nombre(s) de (los) autor(es)\footnote{ID; correo@udlap.mx}
	   \small{Depto. de Actuaría, Física y Matemáticas, Universidad de las Américas Puebla, Puebla, M\'exico 72810}}
\date{\small{\today}}

\maketitle

% ------------------------------------------------------------------------------------------------------------------------------------------------------
% ------------------------------------------------------------------------------------------------------------------------------------------------------
% ------------------------------------------------------------------------------------------------------------------------------------------------------

\begin{abstract}

	soy el abstract
\\
\\
{\it Keywords:}  Angulo, Uniforme, tiempo
\\
\\
\end{abstract}

% ------------------------------------------------------------------------------------------------------------------------------------------------------

\begin{multicols}{2}



\section*{Desarrollo teórico}\label{Desarrollo Teorico}                              	% -------------------- Introduccion
Indispensablemente nuestros objetivos fueron encontrar la aceleración que sufre un cuerpo sujeto a la acción de una fuerza
externa, por el otro lado, predecimos y confrontamos experimentalmente valores de tensión y aceleración en movimientos rectilíneos
de un sistema formado por dos objetos unidos por la misma cuerda que también atraviesa una polea, todo el sistema tenia fricción despreciable.

\section*{Desarrollo Experimental}\label{Desarrollo experimental}				% -------------------- Metodolog'ia

Se monto un riel triangular horizontal sin angulo donde se pudo colocar nuestra masa 1 que consistía de un móvil que pesaba 189.17 gramos y que gracias 
al riel triangular que estaba conectado a un compresor de aire que disminuía la fricción del móvil lo más posible pudimos "reproducir" el fenómeno 
del movimiento uniformemente acelerado, sin embargo, este experimento NO ES REPRODUCIBLE, es decir, no podemos sacar exactamente el valor que obtuvimos en la previa reproducción del experimento,
por lo que sacamos 10 tiempos para cada distancia. Siguiendo con la descripción de nuestro sistema, montamos la polea a un extremo para que de ella pudiese colgar nuestra segunda masa que con un gancho (en donde también consideramos el peso del mismo) se pueden insertar discos de peso,
el peso total de nuestra masa dos fue 149.49 gramos. Terminando con nuestro sistema por ultimo montamos dos compuertas que identificaban un pequeño rectángulo de metal montado en el centro del móvil y estas compuertas se conectaron al smart timer
el cual es el dispositivo que contábamos para medir el tiempo debido a que el smart timer es capaz de conectar con una o dos compuertas para recibir el input de cuando este rectángulo de metal terminaba de pasar por los sensores de movimiento de las fotocompuertas y de esta manera permitirnos 
capturar el tiempo, las fotocompuertas, el compresor de aire y el smart timer estaban conectados a la corriente eléctrica.
\\
Para poder capturar el tiempo primero tuvimos que hacer el test para comprobar que nuestras fotocompuertas efectivamente estuviesen funcionando correctamente y a partir de ese momento con el sistema montado y todo funcionando correctamente empezamos a recopilar nuestros resultados.     
\section*{Resultados y análisis}\label{Resultados}			% -------------------- Resultados


\section*{Conclusiones}\label{Conclusiones}				% -------------------- Conclusiones


\begin{thebibliography}{9}						% -------------------- Bibliograf'ia

%\bibitem{Einstein}
%   Albert Einstein,   \emph{The world as I see it}.   BN %Publishing,   2005.
%
\bibitem{Pérez}
	Pérez H, \emph{Física general 2021}
\end{thebibliography}

% ------------------------------------------------------------------------------------------------------------------------------------------------------
% ------------------------------------------------------------------------------------------------------------------------------------------------------
% ------------------------------------------------------------------------------------------------------------------------------------------------------

\end{multicols}

\end{document}										% -------------------- End Document