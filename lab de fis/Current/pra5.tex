\documentclass{article}
%\usepackage[spanish,activeacute]{babel}
%\usepackage[english,activeacute]{babel}
%\usepackage[latin1]{inputenc}
\usepackage[utf8]{inputenc}
\usepackage[english]{babel}

\usepackage{amsmath,amsfonts,amssymb,amstext,amsthm,amscd}
\usepackage{hyperref}
\usepackage{latexsym}
\usepackage{graphicx}
%\usepackage{subfigure}
\usepackage{subfig}
%\linespread{1.6}
\usepackage{float}
\usepackage{dcolumn}% Align table columns on decimal point(esto lo saque del ejemplo de revtex4)
\usepackage{bm}% bold math(esto lo saque del ejemplo de revtex4)
\newcounter{itemR}
\usepackage{here} %recordar usar el comandao[H] Para las  graficas que es el comando here en lugar de [h!]
\usepackage{fancyhdr}
%\usepackage{sidecap}
%\usepackage[spanish,activeacute]{babel}
\usepackage{multirow}
\usepackage{multicol}
%\usepackage{array}
%\usepackage{booktabs}% para hacer tablas profesionales con \toprule

% ------------------------------------------------------------------------------------------------------------------------------------------------------

\usepackage{fancyhdr}
\setlength{\headheight}{15.2pt}
\usepackage[paperwidth=8.5in, paperheight=11.0in, top=1.0in, bottom=1.0in, left=1.0in, right=1.0in]{geometry}

\pagestyle{fancyplain}
\fancyhead[LE,RO]{Práctica $\#$5, 2nda ley de Newton}
\fancyhead[CE,CO]{}
\fancyhead[RE,LO]{P23-FIS1012-12}
\fancyfoot[LE,RO]{\thepage}
\fancyfoot[CE,CO]{Laboratorio de Física, UDLAP}
\fancyfoot[RE,LO]{}

% ------------------------------------------------------------------------------------------------------------------------------------------------------
% ------------------------------------------------------------------------------------------------------------------------------------------------------
% ------------------------------------------------------------------------------------------------------------------------------------------------------

\begin{document}

\fancypagestyle{plain}{
   	\renewcommand{\headrulewidth}{1pt}
   	\renewcommand{\footrulewidth}{1pt}
}

\renewcommand{\footrulewidth}{1pt}
\renewcommand{\tablename}{Tabla}
\renewcommand{\figurename}{Figura}

% ------------------------------------------------------------------------------------------------------------------------------------------------------
% ------------------------------------------------------------------------------------------------------------------------------------------------------
% ------------------------------------------------------------------------------------------------------------------------------------------------------

\title{Segunda Ley de Newton y movimiento uniformemente acelerado}
\author{\small{Erick Gonzalez Parada ID: 178145, Leonardo Escamilla Salgado ID: 179021 $\&$ Daniela Lomán Barrueta ID: 179062.}\\		% ----- Varios autores separarlos por comas:  \small{Nombre(s) de (los) autor(es)\footnote{ID; correo@udlap.mx}, Nombre(s) de (los) autor(es)\footnote{ID; correo@udlap.mx}
	   \small{Depto. de Actuaría, Física y Matemáticas, Universidad de las Américas Puebla, Puebla, M\'exico 72810}}
\date{\small{\today}}

\maketitle

% ------------------------------------------------------------------------------------------------------------------------------------------------------
% ------------------------------------------------------------------------------------------------------------------------------------------------------
% ------------------------------------------------------------------------------------------------------------------------------------------------------

\begin{abstract}

	soy el abstract
\\
\\
{\it Keywords:}  Polea, Uniforme, tiempo
\\
\\
\end{abstract}

% ------------------------------------------------------------------------------------------------------------------------------------------------------

\begin{multicols}{2}



\section*{Desarrollo teórico}\label{Desarrollo Teorico}                              	% -------------------- Introduccion
La Segunda Ley de Newton establece que la fuerza que actúa sobre un objeto es igual a la masa de ese objeto multiplicada por su aceleración. En otras palabras, la fuerza es la causa del cambio en el movimiento de un objeto. Esta ley se puede expresar matemáticamente como F = ma, donde F es la fuerza neta aplicada a un objeto, m es su masa y a es su aceleración.
\\
\\
Si un objeto está siendo movido por una cuerda con una tensión T en una dirección, mientras que la gravedad está actuando sobre él en otra dirección, se producirá una aceleración en la dirección de la fuerza neta, que es la suma de las fuerzas aplicadas. Esta fuerza neta se puede expresar como:

\begin{equation*}\label{Ec:1}
	Fnet = T - mg	
\end{equation*}

Donde T es la tensión en la cuerda, m es la masa del objeto y g es la aceleración debida a la gravedad.


\begin{figure}[H]
	\centering
	\includegraphics[scale=0.5]{../../static/5fig1.png}
	\caption{Diagrama de cuerpo libre del la masa 1}
	\label{fig:1}
\end{figure}
Para analizar las fuerzas que actúan sobre el objeto, se puede utilizar un diagrama de cuerpo libre. En este diagrama \ref{fig:1}, se muestran todas las fuerzas que actúan sobre la masa 1.

\begin{figure}[H]
	\centering
	\includegraphics[scale=0.5]{../../static/5fig2.png}
	\caption{Diagrama de cuerpo libre del la masa 1}
	\label{fig:2}
\end{figure}
Para el caso de un objeto figura \ref*{fig:2} suspendido de una cuerda en reposo, las fuerzas que actúan sobre el objeto son la fuerza gravitatoria (g) y la fuerza normal (N) de la superficie que soporta el objeto.
\\
\\
La fuerza de la tensión actúa hacia arriba y es igual a la fuerza de la gravedad hacia abajo en este caso, lo que significa que la fuerza neta es cero y el objeto permanece en reposo. Si la fuerza de la tensión es mayor que la fuerza gravitatoria, el objeto se acelerará hacia arriba. Si la fuerza gravitatoria es mayor que la fuerza de la tensión, el objeto se acelerará hacia abajo.
\\
\\
En resumen, la Segunda Ley de Newton es una ley fundamental en la física que establece la relación entre la fuerza, la masa y la aceleración de un objeto. Para analizar las fuerzas que actúan sobre un objeto, se puede utilizar un diagrama de cuerpo libre. Para el caso de un objeto suspendido de una cuerda, las fuerzas que actúan sobre el objeto son la fuerza de la tensión, la fuerza gravitatoria y la fuerza normal. El objeto se mantendrá en reposo si la fuerza de la tensión y la fuerza gravitatoria son iguales. El libro "Philosophiæ Naturalis Principia Mathematica" de Isaac Newton es una fuente de información importante sobre la Segunda Ley de Newton. (Newton, 1687).
\\
\\
Indispensablemente nuestros objetivos fueron encontrar la aceleración que sufre un cuerpo sujeto a la acción de una fuerza
externa, por el otro lado, predecimos y confrontamos experimentalmente valores de tensión y aceleración en movimientos rectilíneos
de un sistema formado por dos objetos unidos por la misma cuerda que también atraviesa una polea, todo el sistema tenia fricción despreciable.

\section*{Desarrollo Experimental}\label{Desarrollo experimental}				% -------------------- Metodolog'ia

\begin{figure}[H]
	\centering
	\includegraphics[scale=0.2]{../../static/5fig3.png}
	\caption*{Sistema montado en el laboratorio en sistema de plano horizontal con fricción despreciable}
	\label{fig:3}
\end{figure}

Se monto un riel triangular horizontal sin angulo donde se pudo colocar nuestra masa 1 que consistía de un móvil que pesaba 189.17 gramos y que gracias 
al riel triangular que estaba conectado a un compresor de aire que disminuía la fricción del móvil lo más posible pudimos "reproducir" el fenómeno 
del movimiento uniformemente acelerado, sin embargo, este experimento NO ES REPRODUCIBLE, es decir, no podemos sacar exactamente el valor que obtuvimos en la previa reproducción del experimento,
por lo que sacamos 10 tiempos para cada distancia. Siguiendo con la descripción de nuestro sistema, montamos la polea a un extremo para que de ella pudiese colgar nuestra segunda masa que con un gancho (en donde también consideramos el peso del mismo) se pueden insertar discos de peso,
el peso total de nuestra masa dos fue 149.49 gramos. Terminando con nuestro sistema por ultimo montamos dos compuertas que identificaban un pequeño rectángulo de metal montado en el centro del móvil y estas compuertas se conectaron al smart timer
el cual es el dispositivo que contábamos para medir el tiempo debido a que el smart timer es capaz de conectar con una o dos compuertas para recibir el input de cuando este rectángulo de metal terminaba de pasar por los sensores de movimiento de las fotocompuertas y de esta manera permitirnos 
capturar el tiempo, las fotocompuertas, el compresor de aire y el smart timer estaban conectados a la corriente eléctrica y atamos con la misma cuerda las masas de tal manera que al soltar la segunda masa se mueva sobre el riel la primera masa (todo lo anterior dicho hace referencia a la figura \ref*{fig:3}).
\\
Para poder capturar el tiempo primero tuvimos que hacer el test para comprobar que nuestras fotocompuertas efectivamente estuviesen funcionando correctamente y a partir de ese momento con el sistema montado y todo funcionando correctamente empezamos a recopilar nuestros resultados.     

\subsection*{Procedimiento experimental}\label{Procedimiento experimental}
Se sostiene la masa 2 para que el móvil no se mueva, se prepara el smart timer para que al soltar la masa 2 el móvil pase por las fotocompuertas y se capture el tiempo.
Para cada distancia se midió 5 veces el tiempo.
\section*{Resultados y análisis}\label{Resultados}			% -------------------- Resultados


\section*{Conclusiones}\label{Conclusiones}				% -------------------- Conclusiones


\begin{thebibliography}{9}						% -------------------- Bibliograf'ia

%\bibitem{Einstein}
%   Albert Einstein,   \emph{The world as I see it}.   BN %Publishing,   2005.
%
\bibitem{Pérez}
	Newton, I. (1687). Philosophiæ Naturalis Principia Mathematica [Mathematical Principles of Natural Philosophy]. Londini: Jussu Societatis Regiæ ac Typis Josephi Streater.
\end{thebibliography}

% ------------------------------------------------------------------------------------------------------------------------------------------------------
% ------------------------------------------------------------------------------------------------------------------------------------------------------
% ------------------------------------------------------------------------------------------------------------------------------------------------------

\end{multicols}

\end{document}										% -------------------- End Document