\documentclass{article}
%\usepackage[spanish,activeacute]{babel}
%\usepackage[english,activeacute]{babel}
%\usepackage[latin1]{inputenc}
\usepackage[utf8]{inputenc}
\usepackage[english]{babel}

\usepackage{amsmath,amsfonts,amssymb,amstext,amsthm,amscd}
\usepackage{hyperref}
\usepackage{latexsym}
\usepackage{graphicx}
%\usepackage{subfigure}
\usepackage{subfig}
%\linespread{1.6}
\usepackage{float}
\usepackage{dcolumn}% Align table columns on decimal point(esto lo saque del ejemplo de revtex4)
\usepackage{bm}% bold math(esto lo saque del ejemplo de revtex4)
\newcounter{itemR}
\usepackage{here} %recordar usar el comando[H] para las gráficas que es el comando here en lugar de [h!]
\usepackage{fancyhdr}
%\usepackage{sidecap}
%\usepackage[spanish,activeacute]{babel}
\usepackage{multirow}
\usepackage{multicol}
\usepackage{array}
%\usepackage{booktabs}% para hacer tablas profesionales con \toprule

% ------------------------------------------------------------------------------------------------------------------------------------------------------

\usepackage{fancyhdr}
\setlength{\headheight}{15.2pt}
\usepackage[paperwidth=8.5in, paperheight=11.0in, top=1.0in, bottom=1.0in, left=1.0in, right=1.0in]{geometry}

\pagestyle{fancyplain}
\fancyhead[LE,RO]{Práctica $\#$5, 2nda ley de Newton}
\fancyhead[CE,CO]{}
\fancyhead[RE,LO]{P23-FIS1012-12}
\fancyfoot[LE,RO]{\thepage}
\fancyfoot[CE,CO]{Laboratorio de Física, UDLAP}
\fancyfoot[RE,LO]{}

% ------------------------------------------------------------------------------------------------------------------------------------------------------
% ------------------------------------------------------------------------------------------------------------------------------------------------------
% ------------------------------------------------------------------------------------------------------------------------------------------------------

\begin{document}

\fancypagestyle{plain}{
   	\renewcommand{\headrulewidth}{1pt}
   	\renewcommand{\footrulewidth}{1pt}
}

\renewcommand{\footrulewidth}{1pt}
\renewcommand{\tablename}{Tabla}
\renewcommand{\figurename}{Figura}

% ------------------------------------------------------------------------------------------------------------------------------------------------------
% ------------------------------------------------------------------------------------------------------------------------------------------------------
% ------------------------------------------------------------------------------------------------------------------------------------------------------

\title{Segunda ley de Newton con plano inclinado}
\author{\small{Luis Alberto Gil Bocanegra ID: 177410, Erick Gonzalez Parada ID: 178145}\\
 \small{Daniela Hernández García ID: 179051 $\&$ Luis Francisco Avila Romero ID: 177632.}\\		% ----- Varios autores separarlos por comas:  \small{Nombre(s) de (los) autor(es)\footnote{ID; correo@udlap.mx}, Nombre(s) de (los) autor(es)\footnote{ID; correo@udlap.mx}
	   \small{Depto. de Actuaría, Física y Matemáticas, Universidad de las Américas Puebla, Puebla, M\'exico 72810}}
\date{\small{\today}}

\maketitle

% ------------------------------------------------------------------------------------------------------------------------------------------------------
% ------------------------------------------------------------------------------------------------------------------------------------------------------
% ------------------------------------------------------------------------------------------------------------------------------------------------------

\begin{abstract}
abstract
\\
\\
{\it Keywords:}  Polea, Plano, Inclinado 
\\
\\
\end{abstract}

% ------------------------------------------------------------------------------------------------------------------------------------------------------

\begin{multicols}{2}

\section{Desarrollo teórico}\label{Desarrollo Teorico}                              	% -------------------- Introducción

\section{Desarrollo Experimental}\label{Desarrollo experimental}				% -------------------- Metodología 

\subsection{Procedimiento experimental}\label{Procedimiento experimental}

\end{multicols}
\section{Resultados y análisis}\label{Resultados}			% -------------------- Resultados
En la tabla \ref{table:1} se muestran los resultados obtenidos durante la fase experimental, siendo la columna Y la de la distancia, mientras que en el eje X, se encuentra el tiempo promedio que tardó el objeto en recorrer dicha distancia con sus respectivas incertidumbres. Igualmente se encuentra el tiempo al cuadrado. Debajo de dicha tabla, se encuentran las incertidumbres calculadas con los mismos datos obtenidos de la tabla (Sy, Sm, Sb). En este caso, la pendiente (M) que es el mismo valor que el de velocidad, dio un resultado de (53.1 $\pm$ 8.5). 

\begin{table}[H]
	\centering
	\begin{tabular}{|c|c|c|}
		\hline
		Distancia(cm) $\pm$ 0.05 & Tiempo(s) & $Tiempo^2$(s) \\
		\hline
		25.0 & 0.3471 $\pm$ 0.03 & 0.1204 $\pm$ 0.00003 \\
		\hline
		30.0 & 0.4175 $\pm$ 0.05 & 0.1743 $\pm$ 0.00005 \\
		\hline
		35.0 & 0.4771 $\pm$ 0.06 & 0.2276 $\pm$ 0.00006 \\
		\hline
		40.0 & 0.5433 $\pm$ 0.02 & 0.2951 $\pm$ 0.00002 \\
		\hline
		45.0 & 0.5856 $\pm$ 0.05 & 0.3429 $\pm$ 0.00005 \\
		\hline
		50.0 & 0.6858 $\pm$ 0.04 & 0.4703 $\pm$ 0.00004 \\
		\hline
		55.0 & 0.8326 $\pm$ 0.02 & 0.6932 $\pm$ 0.00002 \\
		\hline
		60.0 & 0.8631 $\pm$ 0.05 & 0.7449 $\pm$ 0.00005 \\
		\hline
		65.0 & 0.9079 $\pm$ 0.02 & 0.8242 $\pm$ 0.00002 \\
		\hline
		70.0 & 0.9143 $\pm$ 0.01 & 0.8359 $\pm$ 0.00001 \\
		\hline
	\end{tabular}
	\caption{Tabla de valores obtenidos en el experimento}
	\label{table:1}
\end{table}

En la gráfica \ref{fig:1} se muestran los resultados de la tabla \ref{table:1} graficados para su mejor comprensión, como se puede observar, hay ciertos puntos que se encuentran algo separados de la linea de tendencia, es decir, que poseen un rango de error considerable en comparación con la recta ideal, esto debido a diversos factores que se presentaron al momento de la realización del experimento como pudo haber sido la forma en qué se lanzó el móvil, la fuerza implicada en el lanzamiento o alguna anomalía en la superficie que recorre el móvil. De igual forma, se puede observar la ecuación de la recta en la esquina superior izquierda, siendo esta y=(53.1 $\pm$ 8.5)x + 22.3 $\pm$ 29.5, junto con el valor R2, siendo este \emph{R2 = 0.96 }significando que la toma de medidas fue bastante buena, con un margen de error de 0.04. La aceleración del objeto durante su movimiento, que en este caso es el doble de la velocidad, es de 106.2 m/s. Por último, como se puede apreciar, la distancia es proporcional al tiempo, ya que, conforme va aumentando la distancia que recorre el móvil, el tiempo que este tarda en recorrerla aumenta proporcionalmente.

\begin{figure}[H]
	\centering
	\includegraphics[scale=1.2]{../../static/fpra6f1.jpeg}	
	\caption{Representación visual de las tablas de valores}
	\label{fig:1}
\end{figure}

Después tenemos los cálculos realizados analíticamente para obtener la aceleración de los bloques, comparándolas se puede observar que existe una gran diferencia entre ellas, siendo la experimental de 106.2cm/s, mientras que la analítica es de 12.42 m/s, la gran diferencia entre ellas puede ser debido a diversos factores que interfirieron al momento de la realización del experimento, ya sea la distancia a la que fue lanzada, algún tipo de fuerza externa que haya influido en la aceleración del mismo.

\begin{figure}[H]
	\centering
	\includegraphics[scale=0.9]{../../static/fpra6f2.png}
	\caption{Diagrama de cuerpo libre}
	\label{fig:2}
\end{figure}

\begin{multicols}{2}

\begin{equation*}
	N = W 
\end{equation*}


\begin{equation*}
	t - W = ma	
\end{equation*}
%-------------
\begin{equation*}
	t = ma + W
\end{equation*}

\begin{equation*}
t = ma + mg * cos(250^\circ)
\end{equation*}

\begin{equation*}
t = -607 + (-0.67)
\end{equation*}

\begin{equation*}
t= -6.74
\end{equation*}

\begin{equation*}
W = 0.200kg * (-9.81)
\end{equation*}

\begin{equation*}
W = -1.962 
\end{equation*}

Calculando aceleración
\begin{equation*}
	m_1 * a + W_1 = m_2 * a - W_2
\end{equation*}

\begin{equation*}
	m_1 * a - m_2 * a = -W_2 - W_1
\end{equation*}

\begin{equation*}
	a * (m_1 - m_2) = -W_2 - W_1
\end{equation*}

\begin{equation*}
	a = \frac{-W_2 - W_1}{m_1 - m_2}
\end{equation*}

\begin{equation*}
a = \frac{-3.592}{0.289}
\end{equation*}

\begin{equation*}
	a = -12.42
\end{equation*}




\section{Conclusiones}\label{Conclusiones}				% -------------------- Conclusiones

\begin{thebibliography}{9}						% -------------------- Bibliografía

\bibitem{Pérez}
	Newton, I. (1687). Philosophiæ Naturalis Principia Mathematica [Mathematical Principles of Natural Philosophy]. Londini: Jussu Societatis Regiæ ac Typis Josephi Streater.
\end{thebibliography}
\end{multicols}

\end{document}	