\documentclass{article}
%\usepackage[spanish,activeacute]{babel}
%\usepackage[english,activeacute]{babel}
%\usepackage[latin1]{inputenc}
\usepackage[utf8]{inputenc}
\usepackage[english]{babel}

\usepackage{amsmath,amsfonts,amssymb,amstext,amsthm,amscd}
\usepackage{hyperref}
\usepackage{latexsym}
\usepackage{graphicx}
%\usepackage{subfigure}
\usepackage{subfig}
%\linespread{1.6}
\usepackage{float}
\usepackage{dcolumn}% Align table columns on decimal point(esto lo saque del ejemplo de revtex4)
\usepackage{bm}% bold math(esto lo saque del ejemplo de revtex4)
\newcounter{itemR}
\usepackage{here} %recordar usar el comando[H] para las gráficas que es el comando here en lugar de [h!]
\usepackage{fancyhdr}
%\usepackage{sidecap}
%\usepackage[spanish,activeacute]{babel}
\usepackage{multirow}
\usepackage{multicol}
\usepackage{array}
\usepackage{enumitem}
%\usepackage{booktabs}% para hacer tablas profesionales con \toprule

% ------------------------------------------------------------------------------------------------------------------------------------------------------

\usepackage{fancyhdr}
\setlength{\headheight}{15.2pt}
\usepackage[paperwidth=8.5in, paperheight=11.0in, top=1.0in, bottom=1.0in, left=1.0in, right=1.0in]{geometry}

\pagestyle{fancyplain}
\fancyhead[LE,RO]{Práctica $\#$9, colisiones}
\fancyhead[CE,CO]{}
\fancyhead[RE,LO]{P23-FIS1012-12}
\fancyfoot[LE,RO]{\thepage}
\fancyfoot[CE,CO]{Laboratorio de Física, UDLAP}
\fancyfoot[RE,LO]{}

% ------------------------------------------------------------------------------------------------------------------------------------------------------
% ------------------------------------------------------------------------------------------------------------------------------------------------------
% ------------------------------------------------------------------------------------------------------------------------------------------------------

\begin{document}

\fancypagestyle{plain}{
   	\renewcommand{\headrulewidth}{1pt}
   	\renewcommand{\footrulewidth}{1pt}
}

\renewcommand{\footrulewidth}{1pt}
\renewcommand{\tablename}{Tabla}
\renewcommand{\figurename}{Figura}

% ------------------------------------------------------------------------------------------------------------------------------------------------------
% ------------------------------------------------------------------------------------------------------------------------------------------------------
% ------------------------------------------------------------------------------------------------------------------------------------------------------

\title{Colisiones elásticas}
\author{\small{Luis Alberto Gil Bocanegra ID: 177410, Erick Gonzalez Parada ID: 178145}\\
 \small{Daniela Hernández García ID: 179051 $\&$ Luis Francisco Avila Romero ID: 177632.}\\		% ----- Varios autores separarlos por comas:  \small{Nombre(s) de (los) autor(es)\footnote{ID; correo@udlap.mx}, Nombre(s) de (los) autor(es)\footnote{ID; correo@udlap.mx}
	   \small{Depto. de Actuaría, Física y Matemáticas, Universidad de las Américas Puebla, Puebla, M\'exico 72810}}
\date{\small{\today}}

\maketitle

% ------------------------------------------------------------------------------------------------------------------------------------------------------
% ------------------------------------------------------------------------------------------------------------------------------------------------------
% ------------------------------------------------------------------------------------------------------------------------------------------------------

\begin{abstract}
	Esta práctica se basó en demostrar que el momento lineal y la energía cinética
neta se conserva durante la colisión elástica frontal de dos objetos en un sistema con
fricción despreciable, mediante dos móviles en un riel rectangular impulsados por una
fuerza y tomando las velocidades de estos al momento y después del choque, donde al final no se conservo el momento	
\\
\\
{\it Keywords:}  colisión, energía  
\\
\\
\end{abstract}

% ------------------------------------------------------------------------------------------------------------------------------------------------------

\begin{multicols}{2}

\section{Desarrollo teórico}\label{Desarrollo Teorico}                              	% -------------------- Introducción
Nuestro objetivo fue demostrar que el momento lineal y la energía cinética neta se conserva durante la colisión elástica frontal de dos objetos en un sistema con fricción despreciable.
\\
\\
En una colisión elástica, la energía cinética total del sistema se conserva. En otras palabras, la energía no se pierde ni se gana durante la colisión. Esto se cumple siempre y cuando la colisión se dé sin fricción y sin deformación permanente de los cuerpos que colisionan.

Para un sistema de dos cuerpos en una colisión elástica, se puede demostrar que la conservación de la energía cinética y del momento lineal se pueden expresar matemáticamente de la siguiente manera:

\begin{equation}
\frac{1}{2}m_1v_{1i}^2 + \frac{1}{2}m_2v_{2i}^2 = \frac{1}{2}m_1v_{1f}^2 + \frac{1}{2}m_2v_{2f}^2
\end{equation}

\begin{equation}
m_1v_{1i} + m_2v_{2i} = m_1v_{1f} + m_2v_{2f}
\end{equation}

Donde $m_i$ y $v_{i}$ son las masas y velocidades iniciales de los cuerpos, mientras que $m_f$ y $v_{f}$ son las masas y velocidades finales de los cuerpos después de la colisión.

Estas ecuaciones pueden ser resueltas para encontrar las velocidades finales de los cuerpos después de la colisión:

\begin{equation}
v_{1f} = \frac{m_1 - m_2}{m_1 + m_2}v_{1i} + \frac{2m_2}{m_1 + m_2}v_{2i}
\end{equation}

\begin{equation}
v_{2f} = \frac{2m_1}{m_1 + m_2}v_{1i} + \frac{m_2 - m_1}{m_1 + m_2}v_{2i}
\end{equation}

Estas ecuaciones muestran que la velocidad final de un cuerpo depende tanto de su propia masa y velocidad inicial como de la masa y velocidad inicial del otro cuerpo. En otras palabras, los cuerpos intercambian energía y momento durante la colisión.

Además, se puede demostrar que la colisión es elástica si y solo si se cumple que:

\begin{equation}
v_{1i} - v_{2i} = -(v_{1f} - v_{2f})
\end{equation}

Esta ecuación demuestra que la velocidad relativa de los cuerpos antes de la colisión es igual y opuesta a la velocidad relativa después de la colisión en una colisión elástica.

En resumen, en una colisión elástica se conserva tanto la energía cinética total del sistema como el momento lineal, y la velocidad relativa de los cuerpos antes y después de la colisión es igual y opuesta.


\section{Desarrollo Experimental}\label{Desarrollo experimental}				% -------------------- Metodología 
\begin{figure}[H]
	\centering
	\includegraphics[scale=0.5]{../../static/f1.png}
	\caption{sistema armado}
	\label{fig:1}
\end{figure}

Calibración del sistema
\begin{itemize}
\item Colocar el riel triangular sobre la mesa y alinear horizontalmente. Conectar la manguera del compresor al riel para introducir aire y de esta forma reducir la fricción.
\item Pegar con cinta las regletas sobre los móviles, estas servirán de bandera para detección de las velocidades en las fotocompuertas.
\item Colocar las fotocompuertas a distancias tales que puedan medir la velocidad del móvil antes y después de la colisión (fotocompuerta 1 que mida la velocidad del móvil 1 y la fotocompuerta 2 mida la velocidad del móvil 2).
\item Colocar a los móviles el bloque con unas ligas para evitar que no choquen directamente. (Como se muestra en fig1)
\item En el Smart Timer seleccionar Speed —> Collision
\end{itemize}

Procedimiento experimental
\begin{itemize}
\item Lanzar ambos bloques tal que la colisión se lleve en el espacio de las fotocompuertas y detectar medición de ambas velocidades (antes y después del choque) para ambos bloques.
\item Repetir para diferentes configuraciones, al cambiar la velocidad de los bloques (la fuerza de lanzamiento) y las masas.
\end{itemize}

Consideraciones generales
\begin{itemize}
\item Asegurarse que los móviles no choquen directamente uno con el otro o en los extremos del riel.
\item No aplicar demasiada fuerza para evitar que los bloques salgan del riel.
\item Asegurarse que las regletas estén bien puestas y que las fotocompuertas estén a una adecuada altura para detectarlas.
\item Asegurarse que el Smart Timer tome dos velocidades de cada móvil.
\item Observar que los móviles choquen en el espacio entre ambas fotocompuertas para obtener la medición.
\end{itemize}







\end{multicols}
\section{Resultados y análisis}\label{Resultados}			% -------------------- Resultados
Se realizaron 3 mediciones por caso, dichas mediciones se ven representadas en las tablas \ref{tab:colision} y \ref{tab:colision_despues} que se tienen en la parte de abajo del documento, en las cuales hay 2 tablas en las que los móviles tienen la misma cantidad de masa, mientras que las otras 2, las masas son diferentes. Primero el enfoque irá dirigido a cuando las masas de los móviles son iguales, se puede apreciar que, en el primer intento, el móvil 1 de 184g sale con una velocidad de 50.7 cm/s mientras que el móvil 2 de igual magnitud de masa sale con una velocidad de 33.2 cm/s, colisionan y las velocidades de estos cambia considerablemente, siendo de 11.3cm/s la del móvil 1 mientras que la del móvil 2 es de 10.7 cm/s. En el segundo lanzamiento, el móvil 1 de 184g sale con una velocidad de 38.1 cm/s mientras que el móvil 2 de igual magnitud de masa sale con una velocidad de 26.1 cm/s, colisionan y las velocidades de estos cambia considerablemente, siendo de 7.3 cm/s la del móvil 1 mientras que la del móvil 2 es de 0.9 cm/s. Para el último lanzamiento, el móvil 1 de 184g sale con una velocidad de 105.2 cm/s mientras que el móvil 2 de igual magnitud de masa sale con una velocidad de 0.1 cm/s, colisionan y las velocidades de estos cambia considerablemente, siendo de  12.3 cm/s la del móvil 1 mientras que la del móvil 2 es de 2.7 cm/s. Al observar las mediciones realizadas, se hace ver que, normalmente, en este caso en que las masas son iguales, el objeto que inicia con mayor velocidad antes de colisionar, es el mismo que termina con mayor velocidad después de la colisión. 

\begin{table}[H]
\centering
\begin{tabular}{|c|c|c|c|}
\hline
M1 $\pm$ 0.5 & V1 $\pm$ 0.1 & M2 $\pm$ 0.5 & V2 $\pm$ 0.1 \\
\hline
184.0 & 50.7 & 184.0 & 33.2 \\
\hline
184.0 & 38.1 & 184.0 & 26.1 \\
\hline
184.0 & 105.2 & 184.0 & 0.1 \\
\hline
\end{tabular}
\caption{Antes de la colisión}
\label{tab:colision}
\end{table}

\begin{table}[H]
\centering
\begin{tabular}{|c|c|c|c|}
\hline
M1 $\pm$ 0.5 & V1 $\pm$ 0.1 & M2 $\pm$ 0.5 & V2 $\pm$ 0.1 \\
\hline
184.0 & 11.3 & 184.0 & 10.7 \\
\hline
184.0 & 7.3 & 184.0 & 0.9 \\
\hline
184.0 & 12.3 & 184.0 & 2.7 \\
\hline
\end{tabular}
\caption{Después de la colisión}
\label{tab:colision_despues}
\end{table}

Ahora va el caso en el que los móviles poseen masas diferentes tablas \ref{tab:colision2} y \ref{tab:despues_colision}, el móvil 1 posee una masa de 184g mientras que el móvil 2 posee una masa de 218g. En el primer lanzamiento, el móvil 1 de 184g sale con una velocidad de 95.2 cm/s mientras que el móvil 2 de 218g de masa, sale con una velocidad de 48.0 cm/s, colisionan y las velocidades de dichos móviles cambia, siendo de 14.1 cm/s la velocidad con la que resulta en móvil 1, mientras que el móvil 2 resulta ser de 13.3 cm/s. En el segundo lanzamiento, el móvil 1 de 184g sale con una velocidad de 120.4 cm/s mientras que el móvil 2 de 218g de masa, sale con una velocidad de 39.3 cm/s, colisionan y las velocidades de dichos móviles cambia, siendo de 13.0 cm/s la velocidad con la que resulta en móvil 1, mientras que el móvil 2 resulta ser de 12.2 cm/s. Por último, en el tercer lanzamiento el móvil 1 de 184g sale con una velocidad de 59.1 cm/s mientras que el móvil 2 de 218g de masa, sale con una velocidad de 59.8 cm/s, colisionan y las velocidades de dichos móviles cambia, siendo de 13.5 cm/s la velocidad con la que resulta en móvil 1, mientras que el móvil 2 resulta ser de 13.8 cm/s. Al observar las mediciones, se puede ver que, a pesar que el móvil 2 tenía más masa que el móvil 1, el móvil 2 iba bastante más lento que el móvil 1, esta condición resulta en que la mayoría de las velocidades finales sean similares, ya que la falta de masa que hay en el móvil 1, se compensa con la velocidad a la que este colisiona con el móvil 2.

\begin{table}[H]
\centering
\begin{tabular}{|c|c|c|c|}
\hline
M1 $\pm$ 0.5 & V1 $\pm$ 0.1 & M2 $\pm$ 0.5 & V2 $\pm$ 0.1 \\
\hline
184.0 & 95.2 & 218.0 & 48.0 \\
\hline
184.0 & 120.4 & 218.0 & 39.3 \\
\hline
184.0 & 59.1 & 218.0 & 59.8 \\
\hline
\end{tabular}
\caption{Antes de la colisión}
\label{tab:colision2}
\end{table}

\begin{table}[H]
\centering
\begin{tabular}{|c|c|c|c|}
\hline
M1 $\pm$ 0.5 & V1 $\pm$ 0.1 & M2 $\pm$ 0.5 & V2 $\pm$ 0.1 \\
\hline
184.0 & 14.1 & 218.0 & 13.3 \\
\hline
184.0 & 13 & 218.0 & 12.2 \\
\hline
184.0 & 13.5 & 218.0 & 13.8 \\
\hline
\end{tabular}
\caption{Después de la colisión}
\label{tab:despues_colision}
\end{table}

Al calcular la conservación, tanto de momento como de cinética, se puede apreciar que hay una pérdida significativa en ambas situaciones y en cada uno de los casos medidos, esto se debe a una pérdida de momentos por calor al momento del choque entre ambos móviles, ya que al impactar ambos con velocidades diferentes, se crea cierta fricción en un cuestión de microsegundos, esta fricción es suficiente para hacer que exista una pérdida considerable de momento y de cinética, dando como resultado los valores calculados en las tablas \ref{tab:x}, \ref{tab:y}, \ref{tab:w} y \ref{tab:z}.

\begin{table}[H]
\centering
\begin{tabular}{|c|c|c|}
\hline
Antes & = & Después \\
\hline
15437.6 & = & 4048 \\
\hline
11812.8 & = & 1508.8 \\
\hline
19375.2 & = & 2760 \\
\hline
\end{tabular}
\caption{Conservación momento pesos iguales}
\label{tab:x}
\end{table}

\begin{table}[H]
\centering
\begin{tabular}{|c|c|c|}
\hline
Antes & = & Después \\
\hline
 337891.16 & = & 22280.56 \\
\hline
  196219.44 & = & 4977.2 \\
\hline
 1018168.6 & = & 14589.36  \\
\hline
\end{tabular}
\caption{Conservación de momento y velocidad con pesos iguales}
\label{tab:y}
\end{table}

\begin{table}[H]
\centering
\begin{tabular}{|c|c|c|}
\hline
Antes & = & Después \\
\hline
27980.8 & = & 5493.8  \\
\hline
30721 & = & 5051.6  \\
\hline
23910.8 & = & 5492.4 \\
\hline
\end{tabular}
\caption{Conservación de momento con pesos diferentes}
\label{tab:w}
\end{table}

\begin{table}[H]
\centering
\begin{tabular}{|c|c|c|}
\hline
Antes & = & Después  \\
\hline
1084935.68 & = & 37571.53  \\
\hline
1501996.13 & = & 31771.56  \\
\hline
711126.88 & = & 37524.96  \\
\hline
\end{tabular}
\caption{Conservación de velocidad con pesos diferentes}
\label{tab:z}
\end{table}


\section{Conclusiones}\label{Conclusiones}				% -------------------- Conclusiones

En base a los cálculos realizados, se puede concluir que existe una pérdida significativa de momento y energía cinética en ambos escenarios de colisión, tanto cuando los pesos son iguales como cuando son diferentes. Esta pérdida se debe a la fricción generada durante la colisión, lo que resulta en una transferencia de energía que no se conserva en el sistema.

Estos resultados sugieren que la fricción es un factor importante a considerar en cualquier tipo de colisión, ya que puede afectar significativamente la conservación de la energía y el momento. Es importante destacar que esta pérdida de energía no es deseable en la mayoría de las aplicaciones prácticas y debe ser minimizada en la medida de lo posible.
\begin{thebibliography}{9}						% -------------------- Bibliografía
	\bibitem{Martín}
		Martín, I. (2004). Física General
		\bibitem{Serway}
		Serway, R. A., $\&$ Jewett, J. W. (2008). Física para ciencias e ingeniería. (7.a
ed., Vol. 1). CENGAGE Learning.

\bibitem{Pérez}
	Newton, I. (1687). Philosophiæ Naturalis Principia Mathematica [Mathematical Principles of Natural Philosophy]. Londini: Jussu Societatis Regiæ ac Typis Josephi Streater.
\end{thebibliography}

\end{document}	