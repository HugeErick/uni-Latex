% arara: pdflatex
\documentclass{article}
%\usepackage[spanish,activeacute]{babel}
%\usepackage[english,activeacute]{babel}
%\usepackage[latin1]{inputenc}
\usepackage[utf8]{inputenc}
\usepackage[english]{babel}
\usepackage{amsmath,amsfonts,amssymb,amstext,amsthm,amscd}
\usepackage{hyperref}
\usepackage{latexsym}
\usepackage{graphicx}
\usepackage{subfig}
\usepackage{float}
\usepackage{dcolumn}% Align table columns on decimal point(esto lo saque del ejemplo de revtex4)
\usepackage{bm}% bold math(esto lo saque del ejemplo de revtex4)
\newcounter{itemR}
\usepackage{here} \usepackage{fancyhdr}
\usepackage{multirow}
\usepackage{multicol}
\usepackage{array}
\usepackage{ragged2e}

% ------------------------------------------------------------------------------------------------------------------------------------------------------

\usepackage{fancyhdr}
\setlength{\headheight}{15.2pt}
\usepackage[paperwidth=8.5in, paperheight=11.0in, top=1.0in, bottom=1.0in, left=1.0in, right=1.0in]{geometry}

\pagestyle{fancyplain}
\fancyhead[LE,RO]{Reporte $\#$4}
\fancyhead[CE,CO]{}
\fancyhead[RE,LO]{P24-LRT2032-1}
\fancyfoot[LE,RO]{\thepage}
\fancyfoot[CE,CO]{Laboratorio de Circuitos Eléctricos, UDLAP}
\fancyfoot[RE,LO]{}

% ------------------------------------------------------------------------------------------------------------------------------------------------------
% ------------------------------------------------------------------------------------------------------------------------------------------------------
% ------------------------------------------------------------------------------------------------------------------------------------------------------

\begin{document}

\fancypagestyle{plain}{
   	\renewcommand{\headrulewidth}{1pt}
   	\renewcommand{\footrulewidth}{1pt}
}

\renewcommand{\footrulewidth}{1pt}
\renewcommand{\tablename}{Tabla}
\renewcommand{\figurename}{Figura}

% ------------------------------------------------------------------------------------------------------------------------------------------------------
% ------------------------------------------------------------------------------------------------------------------------------------------------------
% ------------------------------------------------------------------------------------------------------------------------------------------------------

\title{Osciloscopio y generador de funciones}
\author{\small{Erick Gonzalez Parada ID: 178145, Jaime Natanael López Valdespino ID: 178039
	$\&$ Santiago Juárez Alvarez }\\ 
\small{ Depto. Computación Electrónica y Mecatrónica.} \\
\small {Docente: Mtro. Omar Fernando Ortiz Aguilera}}
\date{\small{\today}}
\maketitle
% ------------------------------------------------------------------------------------------------------------------------------------------------------

\begin{abstract}
	\begin{justify}
		abstract
		Se aprendió y analizo el osciloscopio y el generador de funciones durante la sesión de laboratorio
		\end{justify}
{\it Keywords:}   función, botones 
\end{abstract}
\begin{multicols}{2}
\section{Objetivo}\label{Objetivo}
\section{Introducción marco teórico}\label{sec:intro}
\subsection{Materiales}\label{sec:materiales}
\subsection{Procedimientos experimentales}\label{sec:materiales}
\section{Resultados}\label{sec:resEsperados}
\subsection{Discusión de los resultados}\label{sec:discus}
\subsection{Comparación teórico práctica}\label{sec:comparacion}
\section{Conclusiones}\label{sec:conclusion}
\section*{Referencias}\label{sec:referencias}	
\end{multicols}
\end{document}

